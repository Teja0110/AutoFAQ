\hypertarget{paul-of-venice}{%
\section{\texorpdfstring{\href{https://plato.stanford.edu/entries/paul-venice/index.html}{Paul
of Venice}}{Paul of Venice}}\label{paul-of-venice}}

\hypertarget{what-does-he-do-with-the-meaning-of-the-termres}{%
\subsection{What does he do with the meaning of the
term'res'?}\label{what-does-he-do-with-the-meaning-of-the-termres}}

\begin{quote}
\emph{summarized\_paragraph} : Paul agrees with Gregory that what is
signified by a simple expression is different from what is Signified by
complex expression. \textbf{But disagrees with him about the meaning of
the term `res' In his view, only positive beings are things, and a
complex expression is not a positive being as it is not an item in one
of the ten categorial lines.} In other words, even though states of
affairs are not really, but only formally, different from individual
things, they cannot in any way be regarded as things in the world.
\end{quote}

\begin{quote}
\emph{avg\_grammar\_rating} : nan\\
\emph{avg\_answerability\_rating} : nan\\
\emph{sum\_yes\_meaningful} : 0\\
\emph{sum\_no\_meaning} : 0\\
\emph{sum\_maybe\_meaning} : 0
\end{quote}

\hypertarget{what-does-paul-deny-that-a-thing-is-a-thing}{%
\subsection{What does paul deny that a thing is a
thing?}\label{what-does-paul-deny-that-a-thing-is-a-thing}}

\begin{quote}
\emph{summarized\_paragraph} : According to Paul, the term `something',
taken in the first or second sense enumerated by Gregory, is a
transcendental term. \textbf{Paul denies that states of affairs, both
actual and possible, are constitutive parts of the world really distinct
from the categorial items, and that whatever is signifiable by a complex
expression is a thing.} If the supporters of Gregory's opinion were to
claim that ` something' is not a transcendent term, it would follow that
`something' would be either less general or more general.
\end{quote}

\begin{quote}
\emph{avg\_grammar\_rating} : nan\\
\emph{avg\_answerability\_rating} : nan\\
\emph{sum\_yes\_meaningful} : 0\\
\emph{sum\_no\_meaning} : 0\\
\emph{sum\_maybe\_meaning} : 0
\end{quote}

\hypertarget{what-do-the-things-have-in-relation-to-the-mind-of-god}{%
\subsection{What do the things have in relation to the mind of
god?}\label{what-do-the-things-have-in-relation-to-the-mind-of-god}}

\begin{quote}
\emph{summarized\_paragraph} : In his commentary on the Categories, Paul
explains that the direct and adequate objects of propositions, which
make them true, are molecular things existing outside the soul. Such
entities are complexe significabilia, the significata of propositions.
\textbf{The first type of truth is the measure of the conformity which
all the things have in relation to their corresponding ideas in the mind
of God, from which they derive.} The second type is also a real property
of extramental things, which measures their various degrees of
disposition to be apprehended by our intellect.
\end{quote}

\begin{quote}
\emph{avg\_grammar\_rating} : nan\\
\emph{avg\_answerability\_rating} : nan\\
\emph{sum\_yes\_meaningful} : 0\\
\emph{sum\_no\_meaning} : 0\\
\emph{sum\_maybe\_meaning} : 0
\end{quote}

\hypertarget{what-is-the-subject-in-a-given-class-of-objects}{%
\subsection{What is the subject in a given class of
objects?}\label{what-is-the-subject-in-a-given-class-of-objects}}

\begin{quote}
\emph{summarized\_paragraph} : Paul builds up a mixed system, where the
copula of the standard philosophical sentences he deals with can have a
threefold value. It means a partial identity between the subject-thing
and the predicate-thing in the case of identical predication; it means a
necessary link between forms in the cases of the first type of formal
predication. \textbf{And it means that the subject in virtue of itself
is necessarily a member of a given class of objects, which the
predication labels and refers to.}
\end{quote}

\begin{quote}
\emph{avg\_grammar\_rating} : nan\\
\emph{avg\_answerability\_rating} : nan\\
\emph{sum\_yes\_meaningful} : 0\\
\emph{sum\_no\_meaning} : 0\\
\emph{sum\_maybe\_meaning} : 0
\end{quote}

\hypertarget{what-is-necessarily-a-member-of-a-given-class-of-objects}{%
\subsection{What is necessarily a member of a given class of
objects?}\label{what-is-necessarily-a-member-of-a-given-class-of-objects}}

\begin{quote}
\emph{summarized\_paragraph} : Paul builds up a mixed system, where the
copula of the standard philosophical sentences he deals with can have a
threefold value. It means a partial identity between the subject-thing
and the predicate-thing in the case of identical predication; it means a
necessary link between forms in the cases of the first type of formal
predication. \textbf{And it means that the subject in virtue of itself
is necessarily a member of a given class of objects, which the
predication labels and refers to.}
\end{quote}

\begin{quote}
\emph{avg\_grammar\_rating} : nan\\
\emph{avg\_answerability\_rating} : nan\\
\emph{sum\_yes\_meaningful} : 0\\
\emph{sum\_no\_meaning} : 0\\
\emph{sum\_maybe\_meaning} : 0
\end{quote}

\hypertarget{what-did-paul-of-venice-develop}{%
\subsection{What did paul of venice
develop?}\label{what-did-paul-of-venice-develop}}

\begin{quote}
\emph{summarized\_paragraph} : Paul of Venice worked out a new version
of the standard medieval realist theories of knowledge, critical of the
most common solutions proper to his intellectual environment. Much more
stress is put on the question of class-membership, and on the
intellective cognition of individuals. \textbf{It is just in order to do
away with the peculiar problems of the realist paradigm that Paul of
Venice developed a new theory of knowledge.} He called it `theory of
knowledge' and said it was a new way of looking at knowledge.
\end{quote}

\begin{quote}
\emph{avg\_grammar\_rating} : nan\\
\emph{avg\_answerability\_rating} : nan\\
\emph{sum\_yes\_meaningful} : 0\\
\emph{sum\_no\_meaning} : 0\\
\emph{sum\_maybe\_meaning} : 0
\end{quote}

\hypertarget{along-with-nardi-ruello-and-what-other-person-who-said-paul-was-an-averroist}{%
\subsection{Along with nardi , ruello , and what other person , who said
paul was an
averroist?}\label{along-with-nardi-ruello-and-what-other-person-who-said-paul-was-an-averroist}}

\begin{quote}
\emph{summarized\_paragraph} : \textbf{According to Nardi, Ruello, and
Kuksewicz, Paul was an Averroist, as he would have supported the thesis
of the unicity and separate character of such an intellect for the whole
human species.} But this is false. On the contrary, Paul's point of view
for this question is close to that of Thomas Aquinas, whose arguments he
utilizes in his commentaries on the De anima and on the Metaphysics. In
these two Aristotelian commentaries he clearly rejects all the main
theses of later medieval A verroism.
\end{quote}

\begin{quote}
\emph{avg\_grammar\_rating} : nan\\
\emph{avg\_answerability\_rating} : nan\\
\emph{sum\_yes\_meaningful} : 0\\
\emph{sum\_no\_meaning} : 0\\
\emph{sum\_maybe\_meaning} : 0
\end{quote}

\hypertarget{what-does-the-fourth-argument-say-that-god-cannot-do}{%
\subsection{What does the fourth argument say that god cannot
do?}\label{what-does-the-fourth-argument-say-that-god-cannot-do}}

\begin{quote}
\emph{summarized\_paragraph} : Paul argues that since in God each
generic principle is different from any corresponding specific principle
, therefore specific principles too are different from each other. He
affirms that if the generic and specific principles were not distinct,
then God could not create something according to the generic principle
without creating it according to a correlated specific principle.
\textbf{The fourth argument is that, since God knows that animal is the
genus of man, He thinks of them by means of two different principles;
otherwise He could not distinguish them.}
\end{quote}

\begin{quote}
\emph{avg\_grammar\_rating} : nan\\
\emph{avg\_answerability\_rating} : nan\\
\emph{sum\_yes\_meaningful} : 0\\
\emph{sum\_no\_meaning} : 0\\
\emph{sum\_maybe\_meaning} : 0
\end{quote}

\hypertarget{how-would-a-general-theory-of-the-proposition-be-more-rigorous}{%
\subsection{How would a general theory of the proposition be more
rigorous?}\label{how-would-a-general-theory-of-the-proposition-be-more-rigorous}}

\begin{quote}
\emph{summarized\_paragraph} : Paul of Venice deals with the problem of
the meaning and truth of sentences in his Logica Magna and in his
commentaries on the Metaphysics and on the Categories. Paul's purpose is
twofold. \textbf{He intends to determine the ontological status and
nature of the complexe significabile more precisely; and to develop a
general theory of the proposition which would be logically more rigorous
and less compromised by a metaphysics of the possible than that
supported by Gregory of Rimini.} For this reason Paul deals with. the
question of the truth and falsity of a proposition before examining the.
problem of its meaning.
\end{quote}

\begin{quote}
\emph{avg\_grammar\_rating} : nan\\
\emph{avg\_answerability\_rating} : nan\\
\emph{sum\_yes\_meaningful} : 0\\
\emph{sum\_no\_meaning} : 0\\
\emph{sum\_maybe\_meaning} : 0
\end{quote}

\hypertarget{the-remote-principle-in-addition-to-the-immanent-principle-is-insufficient-for-causing-the-existence-or-disappearance-of-an-individual}{%
\subsection{The remote principle , in addition to the immanent principle
, is insufficient for causing the existence or disappearance of an
individual?}\label{the-remote-principle-in-addition-to-the-immanent-principle-is-insufficient-for-causing-the-existence-or-disappearance-of-an-individual}}

\begin{quote}
\emph{summarized\_paragraph} : Paul successfully reconciles the
Scotistic approach with certain Thomistic theses. Paul claims that the
principle of individuation is twofold, immanent and remote. The immanent
principle is the one whose presence necessarily entails the existence of
the individual it constitutes. \textbf{The remote principle, on the
other hand, is just what the immanent Principle presupposes, but whose
presence and absence alone are insufficient for causing the existence or
disappearance of an individual.} This means that the relationship
between common natures and singulars is ultimately grounded on
individuated. No actual universality and no instantiation is possible
without individuations.
\end{quote}

\begin{quote}
\emph{avg\_grammar\_rating} : nan\\
\emph{avg\_answerability\_rating} : nan\\
\emph{sum\_yes\_meaningful} : 0\\
\emph{sum\_no\_meaning} : 0\\
\emph{sum\_maybe\_meaning} : 0
\end{quote}

\hypertarget{what-is-the-remote-principle}{%
\subsection{What is the remote
principle?}\label{what-is-the-remote-principle}}

\begin{quote}
\emph{summarized\_paragraph} : Paul successfully reconciles the
Scotistic approach with certain Thomistic theses. Paul claims that the
principle of individuation is twofold, immanent and remote. The immanent
principle is the one whose presence necessarily entails the existence of
the individual it constitutes. \textbf{The remote principle, on the
other hand, is just what the immanent Principle presupposes, but whose
presence and absence alone are insufficient for causing the existence or
disappearance of an individual.} This means that the relationship
between common natures and singulars is ultimately grounded on
individuated. No actual universality and no instantiation is possible
without individuations.
\end{quote}

\begin{quote}
\emph{avg\_grammar\_rating} : nan\\
\emph{avg\_answerability\_rating} : nan\\
\emph{sum\_yes\_meaningful} : 0\\
\emph{sum\_no\_meaning} : 0\\
\emph{sum\_maybe\_meaning} : 0
\end{quote}

\hypertarget{what-is-the-need-for-further-and-higher-activity-on-the-part-of-the-soul}{%
\subsection{What is the need for further and higher activity on the part
of the
soul?}\label{what-is-the-need-for-further-and-higher-activity-on-the-part-of-the-soul}}

\begin{quote}
\emph{summarized\_paragraph} : In Nicoletti's view the phantasm is a
representation of the thing which started the process of knowing. The
intentio singularis vaga still contains a sort of logical description of
the generic matter which the singular thing is made of. For example,
when we apprehend Socrates as a man, we know human essence just as
instantiated by a singular item, and not in itself, as a specific type.
\textbf{In order to have the concept of man, there is need of a further
and higher activity on the part of the soul.} This further activity is
the proper action of the active intellect.
\end{quote}

\begin{quote}
\emph{avg\_grammar\_rating} : nan\\
\emph{avg\_answerability\_rating} : nan\\
\emph{sum\_yes\_meaningful} : 0\\
\emph{sum\_no\_meaning} : 0\\
\emph{sum\_maybe\_meaning} : 0
\end{quote}

\hypertarget{what-is-the-need-for-further-and-higher-activity-on-the-part-of-the-soul-1}{%
\subsection{What is the need for further and higher activity on the part
of the
soul?}\label{what-is-the-need-for-further-and-higher-activity-on-the-part-of-the-soul-1}}

\begin{quote}
\emph{summarized\_paragraph} : In Nicoletti's view the phantasm is a
representation of the thing which started the process of knowing. The
intentio singularis vaga still contains a sort of logical description of
the generic matter which the singular thing is made of. For example,
when we apprehend Socrates as a man, we know human essence just as
instantiated by a singular item, and not in itself, as a specific type.
\textbf{In order to have the concept of man, there is need of a further
and higher activity on the part of the soul.} This further activity is
the proper action of the active intellect.
\end{quote}

\begin{quote}
\emph{avg\_grammar\_rating} : nan\\
\emph{avg\_answerability\_rating} : nan\\
\emph{sum\_yes\_meaningful} : 0\\
\emph{sum\_no\_meaning} : 0\\
\emph{sum\_maybe\_meaning} : 0
\end{quote}

\hypertarget{to-what-do-forms-signified-by-the-subject---term-of-a-proposition-and-the-predicate---term-share-at-least-one-of-their-substrates}{%
\subsection{To what do forms signified by the subject - term of a
proposition and the predicate - term share at least one of their
substrates?}\label{to-what-do-forms-signified-by-the-subject---term-of-a-proposition-and-the-predicate---term-share-at-least-one-of-their-substrates}}

\begin{quote}
\emph{summarized\_paragraph} : \textbf{To speak of identical predication
it is sufficient that the form signified by the subject-term of a
proposition and the formSignified by the predicate-term share at least
one of their substrates of existence.} This is the case for propositions
like `Man is animal' and `The universal-man is something white' (`Homo
in communi est album'). One speaks of formal predication in two cases:
~`The~universal-man~is~something~white' or `He is a man.'
\end{quote}

\begin{quote}
\emph{avg\_grammar\_rating} : nan\\
\emph{avg\_answerability\_rating} : nan\\
\emph{sum\_yes\_meaningful} : 0\\
\emph{sum\_no\_meaning} : 0\\
\emph{sum\_maybe\_meaning} : 0
\end{quote}

\hypertarget{the-essence-of-a-thing-is-different-from-its-being}{%
\subsection{The essence of a thing is different from its
being?}\label{the-essence-of-a-thing-is-different-from-its-being}}

\begin{quote}
\emph{summarized\_paragraph} : Paul: The essence and being of any
creature are not really distinct from each other. \textbf{The essence of
a thing is formally different from its real being and from its essential
being.} Specific and generic essences can keep on being even though no
individual instantiates them, but in this case they do not have any
actual existence. From that point of view, Socrates being a man is in
reality Socrates himself considered together with all the properties of
which he is the bearer. The proposition identifies only one of these
properties, that signified by the predicate-term.
\end{quote}

\begin{quote}
\emph{avg\_grammar\_rating} : nan\\
\emph{avg\_answerability\_rating} : nan\\
\emph{sum\_yes\_meaningful} : 0\\
\emph{sum\_no\_meaning} : 0\\
\emph{sum\_maybe\_meaning} : 0
\end{quote}

\hypertarget{according-to-the-italian-master-what-are-ideas-self---subsistent}{%
\subsection{According to the italian master , what are ideas self -
subsistent?}\label{according-to-the-italian-master-what-are-ideas-self---subsistent}}

\begin{quote}
\emph{summarized\_paragraph} : \textbf{In the commentary on the
Metaphysics, the Italian master, after denying that the ideas are
self-subsistent entities, as Plato thought, maintains that the meaning
of the term `idea' is twofold.} From these definitions, which he
considers consonant with the teaching of Aristotle and Averroes, Paul
derives four consequences or theses, which, taken together, represent
the very core of his theory of divine ideas. Paul: An idea is a specific
essence existing in God's mind as a causal model for the production of
something.
\end{quote}

\begin{quote}
\emph{avg\_grammar\_rating} : nan\\
\emph{avg\_answerability\_rating} : nan\\
\emph{sum\_yes\_meaningful} : 0\\
\emph{sum\_no\_meaning} : 0\\
\emph{sum\_maybe\_meaning} : 0
\end{quote}

\hypertarget{what-does-the-italian-master-do-to-the-idea-of-ideas}{%
\subsection{What does the italian master do to the idea of
ideas?}\label{what-does-the-italian-master-do-to-the-idea-of-ideas}}

\begin{quote}
\emph{summarized\_paragraph} : \textbf{In the commentary on the
Metaphysics, the Italian master, after denying that the ideas are
self-subsistent entities, as Plato thought, maintains that the meaning
of the term `idea' is twofold.} From these definitions, which he
considers consonant with the teaching of Aristotle and Averroes, Paul
derives four consequences or theses, which, taken together, represent
the very core of his theory of divine ideas. Paul: An idea is a specific
essence existing in God's mind as a causal model for the production of
something.
\end{quote}

\begin{quote}
\emph{avg\_grammar\_rating} : nan\\
\emph{avg\_answerability\_rating} : nan\\
\emph{sum\_yes\_meaningful} : 0\\
\emph{sum\_no\_meaning} : 0\\
\emph{sum\_maybe\_meaning} : 0
\end{quote}

\hypertarget{what-is-indirectly-connected-to-the-intelligible-species-by-means-of}{%
\subsection{What is indirectly connected to the intelligible species by
means
of?}\label{what-is-indirectly-connected-to-the-intelligible-species-by-means-of}}

\begin{quote}
\emph{summarized\_paragraph} : In the extramental world universal
essences and individual things come together, since they are really
identical and only formally distinct. In a certain way they are two
different aspects of the same thing. This fact explains what happens in
the process of knowledge according to Paul. \textbf{Since the
intelligible species is directly connected with the essence of a thing
and indirectly, by means of the phantasm, with the singular thing
itself.} Our intellect grasps the universal essence and the individual
thing at the same time, so that both are the primary objects of our
mind.
\end{quote}

\begin{quote}
\emph{avg\_grammar\_rating} : nan\\
\emph{avg\_answerability\_rating} : nan\\
\emph{sum\_yes\_meaningful} : 0\\
\emph{sum\_no\_meaning} : 0\\
\emph{sum\_maybe\_meaning} : 0
\end{quote}

\hypertarget{which-species-is-directly-connected-to-the-essence-of-a-thing}{%
\subsection{Which species is directly connected to the essence of a
thing?}\label{which-species-is-directly-connected-to-the-essence-of-a-thing}}

\begin{quote}
\emph{summarized\_paragraph} : In the extramental world universal
essences and individual things come together, since they are really
identical and only formally distinct. In a certain way they are two
different aspects of the same thing. This fact explains what happens in
the process of knowledge according to Paul. \textbf{Since the
intelligible species is directly connected with the essence of a thing
and indirectly, by means of the phantasm, with the singular thing
itself.} Our intellect grasps the universal essence and the individual
thing at the same time, so that both are the primary objects of our
mind.
\end{quote}

\begin{quote}
\emph{avg\_grammar\_rating} : nan\\
\emph{avg\_answerability\_rating} : nan\\
\emph{sum\_yes\_meaningful} : 0\\
\emph{sum\_no\_meaning} : 0\\
\emph{sum\_maybe\_meaning} : 0
\end{quote}

\hypertarget{what-does-the-direct-and-adequate-objects-of-propositions-make-them}{%
\subsection{What does the direct and adequate objects of propositions
make
them?}\label{what-does-the-direct-and-adequate-objects-of-propositions-make-them}}

\begin{quote}
\emph{summarized\_paragraph} : \textbf{In his commentary on the
Categories, Paul explains that the direct and adequate objects of
propositions, which make them true, are molecular things existing
outside the soul.} Such entities are complexe significabilia, the
significata of propositions. The first type of truth is the measure of
the conformity which all the things have in relation to their
corresponding ideas in the mind of God, from which they derive. The
second type is also a real property of extramental things, which
measures their various degrees of disposition to be apprehended by our
intellect.
\end{quote}

\begin{quote}
\emph{avg\_grammar\_rating} : nan\\
\emph{avg\_answerability\_rating} : nan\\
\emph{sum\_yes\_meaningful} : 0\\
\emph{sum\_no\_meaning} : 0\\
\emph{sum\_maybe\_meaning} : 0
\end{quote}

\hypertarget{according-to-paul-what-do-the-ten-categories-modulate-according-to}{%
\subsection{According to paul , what do the ten categories modulate
according
to?}\label{according-to-paul-what-do-the-ten-categories-modulate-according-to}}

\begin{quote}
\emph{summarized\_paragraph} : \textbf{According to Paul, being is
univocally shared by everything real, since it is the stuff that the ten
categories modulate according to their own essence.} In view of this
position, Paul maintains no real distinction between essence and being.
Like Duns Scotus and Wyclif, Paul speaks of a formal difference (or
difference of reason) between essence and being in creatures. The
essence and the essential being of a thing are one and the same entity
considered from two distinct points of view, intensionally and
extensionally.
\end{quote}

\begin{quote}
\emph{avg\_grammar\_rating} : nan\\
\emph{avg\_answerability\_rating} : nan\\
\emph{sum\_yes\_meaningful} : 0\\
\emph{sum\_no\_meaning} : 0\\
\emph{sum\_maybe\_meaning} : 0
\end{quote}

\hypertarget{according-to-paul-being-is-univocally-shared-by-everything}{%
\subsection{According to paul , being is univocally shared by
everything?}\label{according-to-paul-being-is-univocally-shared-by-everything}}

\begin{quote}
\emph{summarized\_paragraph} : \textbf{According to Paul, being is
univocally shared by everything real, since it is the stuff that the ten
categories modulate according to their own essence.} In view of this
position, Paul maintains no real distinction between essence and being.
Like Duns Scotus and Wyclif, Paul speaks of a formal difference (or
difference of reason) between essence and being in creatures. The
essence and the essential being of a thing are one and the same entity
considered from two distinct points of view, intensionally and
extensionally.
\end{quote}

\begin{quote}
\emph{avg\_grammar\_rating} : nan\\
\emph{avg\_answerability\_rating} : nan\\
\emph{sum\_yes\_meaningful} : 0\\
\emph{sum\_no\_meaning} : 0\\
\emph{sum\_maybe\_meaning} : 0
\end{quote}

\hypertarget{the-second-type-of-extramental-things-measures-the-various-degrees-of-disposition-to-be-apprehended-by-what}{%
\subsection{The second type of extramental things measures the various
degrees of disposition to be apprehended by
what?}\label{the-second-type-of-extramental-things-measures-the-various-degrees-of-disposition-to-be-apprehended-by-what}}

\begin{quote}
\emph{summarized\_paragraph} : In his commentary on the Categories, Paul
explains that the direct and adequate objects of propositions, which
make them true, are molecular things existing outside the soul. Such
entities are complexe significabilia, the significata of propositions.
The first type of truth is the measure of the conformity which all the
things have in relation to their corresponding ideas in the mind of God,
from which they derive. \textbf{The second type is also a real property
of extramental things, which measures their various degrees of
disposition to be apprehended by our intellect.}
\end{quote}

\begin{quote}
\emph{avg\_grammar\_rating} : nan\\
\emph{avg\_answerability\_rating} : nan\\
\emph{sum\_yes\_meaningful} : 0\\
\emph{sum\_no\_meaning} : 0\\
\emph{sum\_maybe\_meaning} : 0
\end{quote}

\hypertarget{the-second-type-of-extramental-things-measures-the-various-degrees-of-disposition-to-be-what-by-our-intellect}{%
\subsection{The second type of extramental things measures the various
degrees of disposition to be what by our
intellect?}\label{the-second-type-of-extramental-things-measures-the-various-degrees-of-disposition-to-be-what-by-our-intellect}}

\begin{quote}
\emph{summarized\_paragraph} : In his commentary on the Categories, Paul
explains that the direct and adequate objects of propositions, which
make them true, are molecular things existing outside the soul. Such
entities are complexe significabilia, the significata of propositions.
The first type of truth is the measure of the conformity which all the
things have in relation to their corresponding ideas in the mind of God,
from which they derive. \textbf{The second type is also a real property
of extramental things, which measures their various degrees of
disposition to be apprehended by our intellect.}
\end{quote}

\begin{quote}
\emph{avg\_grammar\_rating} : nan\\
\emph{avg\_answerability\_rating} : nan\\
\emph{sum\_yes\_meaningful} : 0\\
\emph{sum\_no\_meaning} : 0\\
\emph{sum\_maybe\_meaning} : 0
\end{quote}

\hypertarget{the-soul-is-known-as-what}{%
\subsection{The soul is known as
what?}\label{the-soul-is-known-as-what}}

\begin{quote}
\emph{summarized\_paragraph} : Paul of Venice tried to solve the
aporetic aspects of Duns Scotus' theory of individuation. Scotus said
nothing about the problem of the relation between the thisness and the
particular matter and form that constitute the individual. The
Franciscan master was also silent about a possible identification of the
thiss with one of the two essential forms of the individual substance,
the forma partis (for instance, the individual soul) and the form a
totius. Paul identifies the principle ofindividuation with the informing
act through which the specific nature molds its matter. This
identification had been already suggested in the Summa philosophiae
naturalis.
\end{quote}

\begin{quote}
\emph{avg\_grammar\_rating} : nan\\
\emph{avg\_answerability\_rating} : nan\\
\emph{sum\_yes\_meaningful} : 0\\
\emph{sum\_no\_meaning} : 0\\
\emph{sum\_maybe\_meaning} : 0
\end{quote}

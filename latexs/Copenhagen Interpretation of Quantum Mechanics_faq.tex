\hypertarget{copenhagen-interpretation-of-quantum-mechanics}{%
\section{\texorpdfstring{\href{https://plato.stanford.edu/entries/qm-copenhagen/index.html}{Copenhagen
Interpretation of Quantum
Mechanics}}{Copenhagen Interpretation of Quantum Mechanics}}\label{copenhagen-interpretation-of-quantum-mechanics}}

\hypertarget{what-does-physicists-do-from-their-experiments-on-atomic-objects}{%
\subsection{What does physicists do from their experiments on atomic
objects?}\label{what-does-physicists-do-from-their-experiments-on-atomic-objects}}

\begin{quote}
\emph{summarized\_paragraph} : The aim of Bohr's effort is to give an
empirical interpretation of the quantum formalism. He does not seek to
reduce terms concerning theoretical entities to terms about sense-data
or purely perceptual phenomena. \textbf{He insists only that the
empirical evidence physicists collect from their experiments on atomic
objects has to be described in terms of the same concepts which were
developed in classical mechanics in order for them to understand what
the quantum theory is all about.} There are both similarities and
overlaps between some of the proposed explanations concerning the
indispensability of classical concepts.
\end{quote}

\begin{quote}
\emph{avg\_grammar\_rating} : 4.7\\
\emph{avg\_answerability\_rating} : 4.0\\
\emph{sum\_yes\_meaningful} : 2\\
\emph{sum\_no\_meaning} : 1\\
\emph{sum\_maybe\_meaning} : 0
\end{quote}

\hypertarget{what-theory-states-that-it-is-impossible-to-make-an-unambiguous-separation-between-time-and-space-without-reference-to-the-observer}{%
\subsection{What theory states that it is impossible to make an
unambiguous separation between time and space without reference to the
observer?}\label{what-theory-states-that-it-is-impossible-to-make-an-unambiguous-separation-between-time-and-space-without-reference-to-the-observer}}

\begin{quote}
\emph{summarized\_paragraph} : In general, Bohr considered the demands
of complementarity in quantum mechanics to be logically on a par with
the requirements of relativity in the theory of relativity. He believed
that both theories were a result of novel aspects of the observation
problem, namely the fact that observation in physics is
context-dependent. This again is due to the existence of a maximum
velocity of propagation of all actions in the domain of relativity and a
minimum of any action in theDomain of quantum mechanics. \textbf{And it
is because of these universal limits that it is impossible in the Theory
of relativity to make an unambiguous separation between time and space
without reference to the observer.}
\end{quote}

\begin{quote}
\emph{avg\_grammar\_rating} : 5.0\\
\emph{avg\_answerability\_rating} : 5.0\\
\emph{sum\_yes\_meaningful} : 4\\
\emph{sum\_no\_meaning} : 0\\
\emph{sum\_maybe\_meaning} : 0
\end{quote}

\hypertarget{what-theory-states-that-it-is-impossible-to-make-an-unambiguous-separation-between-time-and-space-without-reference-to-the-observer-1}{%
\subsection{What theory states that it is impossible to make an
unambiguous separation between time and space without reference to the
observer?}\label{what-theory-states-that-it-is-impossible-to-make-an-unambiguous-separation-between-time-and-space-without-reference-to-the-observer-1}}

\begin{quote}
\emph{summarized\_paragraph} : In general, Bohr considered the demands
of complementarity in quantum mechanics to be logically on a par with
the requirements of relativity in the theory of relativity. He believed
that both theories were a result of novel aspects of the observation
problem, namely the fact that observation in physics is
context-dependent. This again is due to the existence of a maximum
velocity of propagation of all actions in the domain of relativity and a
minimum of any action in theDomain of quantum mechanics. \textbf{And it
is because of these universal limits that it is impossible in the Theory
of relativity to make an unambiguous separation between time and space
without reference to the observer.}
\end{quote}

\begin{quote}
\emph{avg\_grammar\_rating} : 5.0\\
\emph{avg\_answerability\_rating} : 5.0\\
\emph{sum\_yes\_meaningful} : 4\\
\emph{sum\_no\_meaning} : 0\\
\emph{sum\_maybe\_meaning} : 0
\end{quote}

\hypertarget{what-should-be-able-to-predict-the-numerical-values-of-plancks-constant}{%
\subsection{What should be able to predict the numerical values of
Planck's
constant?}\label{what-should-be-able-to-predict-the-numerical-values-of-plancks-constant}}

\begin{quote}
\emph{summarized\_paragraph} : The guiding principle behind Bohr's and
later Heisenberg's work in the development of a consistent theory of
atoms was the correspondence rule. The full rule states that a
transition between stationary states is allowed if, and only if, there
is a corresponding harmonic component in the classical motion.
\textbf{The correspondence rule was a heuristic principle meant to make
sure that in areas where the influence of Planck's constant could be
neglected the numerical values predicted by such a theory should be the
same as if they were predicted by classical radiation theory.}
\end{quote}

\begin{quote}
\emph{avg\_grammar\_rating} : 5.0\\
\emph{avg\_answerability\_rating} : 3.8\\
\emph{sum\_yes\_meaningful} : 4\\
\emph{sum\_no\_meaning} : 0\\
\emph{sum\_maybe\_meaning} : 0
\end{quote}

\hypertarget{what-concepts-of-the-classical-theories-will-never-become-superfluous-for-the-description-of-physical-experience}{%
\subsection{What concepts of the classical theories will never become
superfluous for the description of physical
experience?}\label{what-concepts-of-the-classical-theories-will-never-become-superfluous-for-the-description-of-physical-experience}}

\begin{quote}
\emph{summarized\_paragraph} : \textbf{``No more is it likely that the
fundamental concepts of the classical theories will ever become
superfluous for the description of physical experience,'' he says.} ``It
continues to be the application of these concepts alone that makes it
possible to relate the symbolism of the quantum theory to the data of
experience'' ``It is possible to connect the symbolism~of the quantum
theories to thedata of experience . . . and to make it possible for us
to understand the nature of the universe,'' he adds. ``This is the key
to our understanding of the world.''
\end{quote}

\begin{quote}
\emph{avg\_grammar\_rating} : 5.0\\
\emph{avg\_answerability\_rating} : 4.8\\
\emph{sum\_yes\_meaningful} : 3\\
\emph{sum\_no\_meaning} : 1\\
\emph{sum\_maybe\_meaning} : 0
\end{quote}

\hypertarget{according-to-popper-the-fundamental-concepts-of-classical-theories-will-never-become-superfluous-for-the-description-of-what-kind-of-experience}{%
\subsection{According to Popper , the fundamental concepts of classical
theories will never become superfluous for the description of what kind
of
experience?}\label{according-to-popper-the-fundamental-concepts-of-classical-theories-will-never-become-superfluous-for-the-description-of-what-kind-of-experience}}

\begin{quote}
\emph{summarized\_paragraph} : \textbf{``No more is it likely that the
fundamental concepts of the classical theories will ever become
superfluous for the description of physical experience,'' he says.} ``It
continues to be the application of these concepts alone that makes it
possible to relate the symbolism of the quantum theory to the data of
experience'' ``It is possible to connect the symbolism~of the quantum
theories to thedata of experience . . . and to make it possible for us
to understand the nature of the universe,'' he adds. ``This is the key
to our understanding of the world.''
\end{quote}

\begin{quote}
\emph{avg\_grammar\_rating} : 4.0\\
\emph{avg\_answerability\_rating} : 3.7\\
\emph{sum\_yes\_meaningful} : 2\\
\emph{sum\_no\_meaning} : 1\\
\emph{sum\_maybe\_meaning} : 0
\end{quote}

\hypertarget{what-kind-of-physics-must-be-used-in-a-functional-description-of-a-system}{%
\subsection{What kind of physics must be used in a functional
description of a
system?}\label{what-kind-of-physics-must-be-used-in-a-functional-description-of-a-system}}

\begin{quote}
\emph{summarized\_paragraph} : If a measuring instrument is to serve its
purpose of furnishing us with knowledge of an object -- that is to say,
if it is to be described functionally -- it must be described
classically. Of course, it is always possible to represent the
experimental apparatus from a purely structural point of view as a
quantum-mechanical system without any reference to its function.
\textbf{But any functional description in which it is treated as a means
to an end, and not merely as a dynamical system, must use the concepts
of classical physics.}
\end{quote}

\begin{quote}
\emph{avg\_grammar\_rating} : 4.8\\
\emph{avg\_answerability\_rating} : 5.0\\
\emph{sum\_yes\_meaningful} : 4\\
\emph{sum\_no\_meaning} : 0\\
\emph{sum\_maybe\_meaning} : 0
\end{quote}

\hypertarget{what-type-of-description-is-used-to-describe-our-environment}{%
\subsection{What type of description is used to describe our
environment?}\label{what-type-of-description-is-used-to-describe-our-environment}}

\begin{quote}
\emph{summarized\_paragraph} : Bohr thought that properties like
momentum, position, and duration could be attributed only to an atom
object in relation to a specific experimental arrangement. As Dieks
mentions while denying any deeper philosophical motivation on Bohr's
part: the use of classical concepts is part of the laboratory life.
\textbf{``This classical description is basically just the description
in terms of everyday language, generalized by the addition of physics
terminology, and it is the one we de facto use to describe our
environment''}
\end{quote}

\begin{quote}
\emph{avg\_grammar\_rating} : 4.5\\
\emph{avg\_answerability\_rating} : 4.5\\
\emph{sum\_yes\_meaningful} : 3\\
\emph{sum\_no\_meaning} : 1\\
\emph{sum\_maybe\_meaning} : 0
\end{quote}

\hypertarget{what-type-of-concepts-did-not-change}{%
\subsection{What type of concepts did not
change?}\label{what-type-of-concepts-did-not-change}}

\begin{quote}
\emph{summarized\_paragraph} : Bohr believed not just retrospectively
that quantum mechanics was a natural generalization of classical
physics, but he and Heisenberg followed in practice the requirements of
the correspondence rule. \textbf{In the mind of Bohr, the meaning of the
classical concepts did not change but their application was restricted.}
This was the lesson of complementarity. Bohr's practical methodology
stands therefore in direct opposition to Thomas Kuhn and Paul
Feyerabend's historical view that succeeding theories, like classical
mechanics and quantum mechanics, are incommensurable.
\end{quote}

\begin{quote}
\emph{avg\_grammar\_rating} : 5.0\\
\emph{avg\_answerability\_rating} : 2.7\\
\emph{sum\_yes\_meaningful} : 0\\
\emph{sum\_no\_meaning} : 3\\
\emph{sum\_maybe\_meaning} : 0
\end{quote}

\hypertarget{what-was-restricted-to-the-concept-of-classical-concepts}{%
\subsection{What was restricted to the concept of classical
concepts?}\label{what-was-restricted-to-the-concept-of-classical-concepts}}

\begin{quote}
\emph{summarized\_paragraph} : Bohr believed not just retrospectively
that quantum mechanics was a natural generalization of classical
physics, but he and Heisenberg followed in practice the requirements of
the correspondence rule. \textbf{In the mind of Bohr, the meaning of the
classical concepts did not change but their application was restricted.}
This was the lesson of complementarity. Bohr's practical methodology
stands therefore in direct opposition to Thomas Kuhn and Paul
Feyerabend's historical view that succeeding theories, like classical
mechanics and quantum mechanics, are incommensurable.
\end{quote}

\begin{quote}
\emph{avg\_grammar\_rating} : 5.0\\
\emph{avg\_answerability\_rating} : 3.3\\
\emph{sum\_yes\_meaningful} : 3\\
\emph{sum\_no\_meaning} : 0\\
\emph{sum\_maybe\_meaning} : 0
\end{quote}

\hypertarget{what-did-quantum-theory-need-to-be-found}{%
\subsection{What did quantum theory need to be
found?}\label{what-did-quantum-theory-need-to-be-found}}

\begin{quote}
\emph{summarized\_paragraph} : spin is a quantum property of the
electrons which cannot be understood as a classical angular momentum.
Needless to say, Bohr fully understood that. \textbf{But he didn't think
that this discovery ruled out the use of the correspondence rule as
guidance to finding a satisfactory quantum theory.} A lengthy quotation
from Bohr's paper ``The Causality Problem in Atomic Physics'' gives
evidence for this. It says that Bohr ``understood that the theory of
spin is not a theory of angular momentum''.
\end{quote}

\begin{quote}
\emph{avg\_grammar\_rating} : 3.7\\
\emph{avg\_answerability\_rating} : 2.7\\
\emph{sum\_yes\_meaningful} : 1\\
\emph{sum\_no\_meaning} : 2\\
\emph{sum\_maybe\_meaning} : 0
\end{quote}

\hypertarget{what-concepts-were-essential-for-physicists-to-understand-the-nature-of-the-universe}{%
\subsection{What concepts were essential for physicists to understand
the nature of the
universe?}\label{what-concepts-were-essential-for-physicists-to-understand-the-nature-of-the-universe}}

\begin{quote}
\emph{summarized\_paragraph} : Bohr saw the classical concepts as
necessary for procuring unambiguous communication about what happens in
the laboratory. Classical concepts are indispensable, because they
enable physicists to describe observations in a clear common language,
and because they are the ones by which the physicists connect the
mathematical formalism with observational content. \textbf{Bohr wrote in
his book, The Theory of Relativity, that classical concepts are
essential for physicists to understand the nature of the universe and to
make sense of it.} The book is published by Hodder \& Stoughton at
£16.99.
\end{quote}

\begin{quote}
\emph{avg\_grammar\_rating} : 5.0\\
\emph{avg\_answerability\_rating} : 4.3\\
\emph{sum\_yes\_meaningful} : 4\\
\emph{sum\_no\_meaning} : 0\\
\emph{sum\_maybe\_meaning} : 0
\end{quote}

\hypertarget{what-did-bohr-do-in-his-book-the-theory-of-relativity}{%
\subsection{What did bohr do in his book , the theory of
relativity?}\label{what-did-bohr-do-in-his-book-the-theory-of-relativity}}

\begin{quote}
\emph{summarized\_paragraph} : Bohr saw the classical concepts as
necessary for procuring unambiguous communication about what happens in
the laboratory. Classical concepts are indispensable, because they
enable physicists to describe observations in a clear common language,
and because they are the ones by which the physicists connect the
mathematical formalism with observational content. \textbf{Bohr wrote in
his book, The Theory of Relativity, that classical concepts are
essential for physicists to understand the nature of the universe and to
make sense of it.} The book is published by Hodder \& Stoughton at
£16.99.
\end{quote}

\begin{quote}
\emph{avg\_grammar\_rating} : 5.0\\
\emph{avg\_answerability\_rating} : 3.7\\
\emph{sum\_yes\_meaningful} : 3\\
\emph{sum\_no\_meaning} : 0\\
\emph{sum\_maybe\_meaning} : 0
\end{quote}

\hypertarget{who-believed-that-the-wave-function-formalism-is-a-mere-tool-for-prediction}{%
\subsection{Who believed that the wave function formalism is a mere tool
for
prediction?}\label{who-believed-that-the-wave-function-formalism-is-a-mere-tool-for-prediction}}

\begin{quote}
\emph{summarized\_paragraph} : Dennis Dieks argues that Bohr's lecture
was meant to promulgate an instrumentalist interpretation of quantum
theory. \textbf{Dieks goes against the more general interpretation of
Bohr according to which Bohr only believed that the wave function
formalism is a mere tool for prediction.} Just because Bohr writes off
quantum formalism as a pictoral representation, it still gives us some
insight into physical reality, Dieks says. The argument concerns the
fact that theWave function in quantum mechanics cannot represent a
three-dimensional entity.
\end{quote}

\begin{quote}
\emph{avg\_grammar\_rating} : 5.0\\
\emph{avg\_answerability\_rating} : 3.7\\
\emph{sum\_yes\_meaningful} : 3\\
\emph{sum\_no\_meaning} : 0\\
\emph{sum\_maybe\_meaning} : 0
\end{quote}

\hypertarget{what-happens-to-position-or-momentum-talk-over-to-a-quantum-object}{%
\subsection{What happens to position or momentum talk over to a quantum
object?}\label{what-happens-to-position-or-momentum-talk-over-to-a-quantum-object}}

\begin{quote}
\emph{summarized\_paragraph} : Bohr believed that quantum mechanics is
universal in the sense that Heisenberg's indeterminacy relation applies
to both micro- and macroscopic systems due to the quantum of action.
Bohr believed for epistemic reasons that we had to use classical
language because this language is a refinement of our everyday language,
which is adapted to describe our sensory experience. \textbf{Dieks
concludes that the interaction between the measuring device and the
quantum object determines, in the classical textbook examples, whether
position or momentum talk can be carried over to quantum object.}
\end{quote}

\begin{quote}
\emph{avg\_grammar\_rating} : 3.0\\
\emph{avg\_answerability\_rating} : 2.8\\
\emph{sum\_yes\_meaningful} : 1\\
\emph{sum\_no\_meaning} : 3\\
\emph{sum\_maybe\_meaning} : 0
\end{quote}

\hypertarget{what-is-the-symbolism-of-quantum-theories-able-to-relate-to}{%
\subsection{What is the symbolism of quantum theories able to relate
to?}\label{what-is-the-symbolism-of-quantum-theories-able-to-relate-to}}

\begin{quote}
\emph{summarized\_paragraph} : ``No more is it likely that the
fundamental concepts of the classical theories will ever become
superfluous for the description of physical experience,'' he says.
**``It continues to be the application of these concepts alone that
makes it possible to relate the symbolism of the quantum theory to the
data of experience'' ``It is possible to connect the symbolism~of the
quantum theories to thedata of experience .** . . and to make it
possible for us to understand the nature of the universe,'' he adds.
``This is the key to our understanding of the world.''
\end{quote}

\begin{quote}
\emph{avg\_grammar\_rating} : 4.7\\
\emph{avg\_answerability\_rating} : 5.0\\
\emph{sum\_yes\_meaningful} : 3\\
\emph{sum\_no\_meaning} : 0\\
\emph{sum\_maybe\_meaning} : 0
\end{quote}

\hypertarget{what-is-the-symbolism-of-quantum-theories-able-to-relate-to-1}{%
\subsection{What is the symbolism of quantum theories able to relate
to?}\label{what-is-the-symbolism-of-quantum-theories-able-to-relate-to-1}}

\begin{quote}
\emph{summarized\_paragraph} : ``No more is it likely that the
fundamental concepts of the classical theories will ever become
superfluous for the description of physical experience,'' he says.
**``It continues to be the application of these concepts alone that
makes it possible to relate the symbolism of the quantum theory to the
data of experience'' ``It is possible to connect the symbolism~of the
quantum theories to thedata of experience .** . . and to make it
possible for us to understand the nature of the universe,'' he adds.
``This is the key to our understanding of the world.''
\end{quote}

\begin{quote}
\emph{avg\_grammar\_rating} : 5.0\\
\emph{avg\_answerability\_rating} : 5.0\\
\emph{sum\_yes\_meaningful} : 3\\
\emph{sum\_no\_meaning} : 0\\
\emph{sum\_maybe\_meaning} : 0
\end{quote}

\hypertarget{what-did-he-believe-a-phenomenon-to-be-a-measurement-of}{%
\subsection{What did he believe a phenomenon to be a measurement
of?}\label{what-did-he-believe-a-phenomenon-to-be-a-measurement-of}}

\begin{quote}
\emph{summarized\_paragraph} : After EPR paper Bohr spoke about
Heisenberg's ``indeterminacy relation' as indicating the ontological
consequences of his claim that kinematic and dynamic variables are
ill-defined unless they refer to an experimental outcome. Bohr no longer
mentioned descriptions as being complementary, but rather phenomena or
information. He introduced the definition of a ``phenomenon'' as
requiring a complete description of the entire experimental arrangement.
\textbf{He took a phenomenon to be a measurement of the values of either
kinematics or dynamic properties.}
\end{quote}

\begin{quote}
\emph{avg\_grammar\_rating} : 4.3\\
\emph{avg\_answerability\_rating} : 4.0\\
\emph{sum\_yes\_meaningful} : 2\\
\emph{sum\_no\_meaning} : 0\\
\emph{sum\_maybe\_meaning} : 1
\end{quote}

\hypertarget{bohr-did-not-think-of-what-kind-of-measurement}{%
\subsection{bohr did not think of what kind of
measurement?}\label{bohr-did-not-think-of-what-kind-of-measurement}}

\begin{quote}
\emph{summarized\_paragraph} : Howard: Bohr's interpretation of
complementarity and the textbook Copenhagen interpretation
(i.e.~wave-particle duality and wave packet collapse) are incompatible.
\textbf{Henderson: On one side of the spectrum there is Bohr who did not
think of quantum measurement in terms of a collapse of the wave
function.} In the middle we find Heisenberg talking about the collapse
as an objective physical process but thinking that this couldn't be
analyzed any further because of its indeterministic nature.
\end{quote}

\begin{quote}
\emph{avg\_grammar\_rating} : 4.0\\
\emph{avg\_answerability\_rating} : 3.3\\
\emph{sum\_yes\_meaningful} : 2\\
\emph{sum\_no\_meaning} : 2\\
\emph{sum\_maybe\_meaning} : 0
\end{quote}

\hypertarget{according-to-von-neumann-what-does-it-make-to-compare-the-numerical-values-of-the-theory-of-atoms-with-classical-physics}{%
\subsection{According to Von Neumann , what does it make to compare the
numerical values of the theory of atoms with classical
physics?}\label{according-to-von-neumann-what-does-it-make-to-compare-the-numerical-values-of-the-theory-of-atoms-with-classical-physics}}

\begin{quote}
\emph{summarized\_paragraph} : The correspondence rule was based on the
idea that classical concepts were indispensable for our understanding of
physical reality. It is only when classical phenomena and quantum
phenomena are described in terms of the same classical concepts that we
can compare different physical experiences. Bohr directly mentioned the
relationship between the use of classical concepts and the
correspondence principle in 1934 when he wrote in the Introduction to
Atomic Theory and the Description of Nature. \textbf{He wrote: ``It is
obvious that it makes no sense to compare the numerical values of the
theory of atoms with those of classical physics''}
\end{quote}

\begin{quote}
\emph{avg\_grammar\_rating} : 3.0\\
\emph{avg\_answerability\_rating} : 2.7\\
\emph{sum\_yes\_meaningful} : 0\\
\emph{sum\_no\_meaning} : 2\\
\emph{sum\_maybe\_meaning} : 1
\end{quote}

\hypertarget{bohrs-view-was-that-truth-conditions-of-sentences-ascribing-a-certain-kinematic-or-dynamic-value-to-an-atomic-object-are-dependent-on-what-apparatus}{%
\subsection{bohr's view was that truth conditions of sentences ascribing
a certain kinematic or dynamic value to an atomic object are dependent
on what
apparatus?}\label{bohrs-view-was-that-truth-conditions-of-sentences-ascribing-a-certain-kinematic-or-dynamic-value-to-an-atomic-object-are-dependent-on-what-apparatus}}

\begin{quote}
\emph{summarized\_paragraph} : Complementarity is first and foremost a
semantic and epistemological reading of quantum mechanics that carries
certain ontological implications. \textbf{Bohr's view was, to phrase it
in a modern philosophical jargon, that the truth conditions of sentences
ascribing a certain kinematic or dynamic value to an atomic object are
dependent on the apparatus involved.} These truth conditions have to
include reference to the experimental setup as well as the actual
outcome of the experiment. This claim is called Bohr's indefinability
thesis.
\end{quote}

\begin{quote}
\emph{avg\_grammar\_rating} : 3.3\\
\emph{avg\_answerability\_rating} : 2.3\\
\emph{sum\_yes\_meaningful} : 0\\
\emph{sum\_no\_meaning} : 3\\
\emph{sum\_maybe\_meaning} : 0
\end{quote}

\hypertarget{who-was-aware-of-the-idea-of-decoherence}{%
\subsection{Who was aware of the idea of
decoherence?}\label{who-was-aware-of-the-idea-of-decoherence}}

\begin{quote}
\emph{summarized\_paragraph} : Quantum fundamentalists must be ready to
explain why the macroscopic world appears classical. An alternative to
von Neumann's projection postulate is the claim that the formalism
should be read literally and that measurements do not describe the world
as it really is. \textbf{If Bohr had known the idea of decoherence, he
would probably have had no objection to it, as several authors have
pointed to it as a natural dynamical extension of his view that
measurements is an irreversible amplification process.}
\end{quote}

\begin{quote}
\emph{avg\_grammar\_rating} : 5.0\\
\emph{avg\_answerability\_rating} : 2.3\\
\emph{sum\_yes\_meaningful} : 3\\
\emph{sum\_no\_meaning} : 0\\
\emph{sum\_maybe\_meaning} : 0
\end{quote}

\hypertarget{according-to-popper-what-type-of-experience-does-he-believe-there-is-a-physical-correlate-to}{%
\subsection{According to Popper , what type of experience does he
believe there is a physical correlate
to?}\label{according-to-popper-what-type-of-experience-does-he-believe-there-is-a-physical-correlate-to}}

\begin{quote}
\emph{summarized\_paragraph} : In 1932 {[}1996{]}, von Neumann suggested
that the entangled state of the object and the instrument collapses to a
determinate state whenever a measurement takes place. This measurement
process (a type 1-process as he called it) could not be described by
quantum mechanics. He argues that during a measurement the actual
observer gets a subjective perception of what is going on that has a
non-physical nature. \textbf{However, he holds on to psycho-physical
parallelism as a scientific principle, which he interprets such that
there exists a physical correlate to any extra-physical process of the
subjective experience.}
\end{quote}

\begin{quote}
\emph{avg\_grammar\_rating} : 4.3\\
\emph{avg\_answerability\_rating} : 5.0\\
\emph{sum\_yes\_meaningful} : 3\\
\emph{sum\_no\_meaning} : 0\\
\emph{sum\_maybe\_meaning} : 0
\end{quote}

\hypertarget{what-does-he-do-with-psycho---physical-parallelism}{%
\subsection{What does he do with psycho - physical
parallelism?}\label{what-does-he-do-with-psycho---physical-parallelism}}

\begin{quote}
\emph{summarized\_paragraph} : In 1932 {[}1996{]}, von Neumann suggested
that the entangled state of the object and the instrument collapses to a
determinate state whenever a measurement takes place. This measurement
process (a type 1-process as he called it) could not be described by
quantum mechanics. He argues that during a measurement the actual
observer gets a subjective perception of what is going on that has a
non-physical nature. \textbf{However, he holds on to psycho-physical
parallelism as a scientific principle, which he interprets such that
there exists a physical correlate to any extra-physical process of the
subjective experience.}
\end{quote}

\begin{quote}
\emph{avg\_grammar\_rating} : 5.0\\
\emph{avg\_answerability\_rating} : 4.3\\
\emph{sum\_yes\_meaningful} : 1\\
\emph{sum\_no\_meaning} : 1\\
\emph{sum\_maybe\_meaning} : 1
\end{quote}

\hypertarget{what-kind-of-outcomes-are-represented-by-the-wave-function-symbolically}{%
\subsection{What kind of outcomes are represented by the wave function
symbolically?}\label{what-kind-of-outcomes-are-represented-by-the-wave-function-symbolically}}

\begin{quote}
\emph{summarized\_paragraph} : The term ``pictorial representation''
stands for a representation that helps us to visualize what it
represents. A pictorial representation is a formalism that has an
isomorphic relation to the objects it represents such that the
visualized structure of the representation corresponds to a similar
structure in nature. Conversely, a symbolic representation does not
stand for anything visualizable. It is an abstract tool whose function
it is to calculate a result whenever this representation is applied to
an experimental situation. \textbf{With respect to the formalism of
quantum mechanics it is particularly one's interpretation of the wave
function that determines whether one thinks of it symbolically as a tool
for calculation of statistical outcomes.}
\end{quote}

\begin{quote}
\emph{avg\_grammar\_rating} : 5.0\\
\emph{avg\_answerability\_rating} : 3.8\\
\emph{sum\_yes\_meaningful} : 4\\
\emph{sum\_no\_meaning} : 0\\
\emph{sum\_maybe\_meaning} : 0
\end{quote}

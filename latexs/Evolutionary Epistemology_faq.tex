\hypertarget{evolutionary-epistemology}{%
\section{\texorpdfstring{\href{https://plato.stanford.edu/entries/epistemology-evolutionary/index.html}{Evolutionary
Epistemology}}{Evolutionary Epistemology}}\label{evolutionary-epistemology}}

\hypertarget{what-does-an-evolutionary-approach-contribute-to-the-resolution-of-traditional-problems}{%
\subsection{What does an evolutionary approach contribute to the
resolution of traditional
problems?}\label{what-does-an-evolutionary-approach-contribute-to-the-resolution-of-traditional-problems}}

\begin{quote}
\emph{summarized\_paragraph} : There are three possible configurations
of the relationship between descriptive and traditional epistemologies.
Descriptive~epistemologies can be construed as competitors to
traditional normative epistemology. On this view, both are trying to
address the same concerns and offering competing solutions. \textbf{The
extent to which an evolutionary approach contributes to the resolution
of traditional problems is a function of which approach one adopts.} The
evolutionary analyses serve to rule out normative approaches which are
either implausible or inconsistent with an evolutionary origin of human
understanding.
\end{quote}

\begin{quote}
\emph{avg\_grammar\_rating} : 4.7\\
\emph{avg\_answerability\_rating} : 4.7\\
\emph{sum\_yes\_meaningful} : 3\\
\emph{sum\_no\_meaning} : 0\\
\emph{sum\_maybe\_meaning} : 0
\end{quote}

\hypertarget{how-does-an-evolutionary-approach-contribute-to-the-resolution-of-traditional-problems}{%
\subsection{How does an evolutionary approach contribute to the
resolution of traditional
problems?}\label{how-does-an-evolutionary-approach-contribute-to-the-resolution-of-traditional-problems}}

\begin{quote}
\emph{summarized\_paragraph} : There are three possible configurations
of the relationship between descriptive and traditional epistemologies.
Descriptive~epistemologies can be construed as competitors to
traditional normative epistemology. On this view, both are trying to
address the same concerns and offering competing solutions. \textbf{The
extent to which an evolutionary approach contributes to the resolution
of traditional problems is a function of which approach one adopts.} The
evolutionary analyses serve to rule out normative approaches which are
either implausible or inconsistent with an evolutionary origin of human
understanding.
\end{quote}

\begin{quote}
\emph{avg\_grammar\_rating} : 5.0\\
\emph{avg\_answerability\_rating} : 4.3\\
\emph{sum\_yes\_meaningful} : 4\\
\emph{sum\_no\_meaning} : 0\\
\emph{sum\_maybe\_meaning} : 0
\end{quote}

\hypertarget{what-kind-of-phenomenon-has-evolutionary-epistemology-been-concerned-with}{%
\subsection{What kind of phenomenon has evolutionary epistemology been
concerned
with?}\label{what-kind-of-phenomenon-has-evolutionary-epistemology-been-concerned-with}}

\begin{quote}
\emph{summarized\_paragraph} : Every scientific enterprise requires
formal and semi-formal models which allow the quantitative
characterization of its objects of study. The attempt to transform the
philosophical study of knowledge into a scientific discipline which
approaches knowledge as a biological phenomenon is no different.
\textbf{Much of the evolutionary epistemology literature has been
concerned with how to conceive ofknowledge as a natural phenomenon, what
difference this would make to our understanding of our place in the
world, and with answering objections to the project.} There are, as
well, a number of more technical projects which attempt to provide the
theoretical tools necessary for a naturalistic worldview.
\end{quote}

\begin{quote}
\emph{avg\_grammar\_rating} : 5.0\\
\emph{avg\_answerability\_rating} : 4.3\\
\emph{sum\_yes\_meaningful} : 3\\
\emph{sum\_no\_meaning} : 0\\
\emph{sum\_maybe\_meaning} : 0
\end{quote}

\hypertarget{what-is-the-thumb-in-humans-called}{%
\subsection{What is the thumb in humans
called?}\label{what-is-the-thumb-in-humans-called}}

\begin{quote}
\emph{summarized\_paragraph} : \textbf{The development of specific
traits, such as the opposable thumb in humans, can be viewed both from
the point of view of the development of that trait in individual
organisms.} The development of knowledge and knowing mechanisms exhibits
a parallel distinction. One might expect that since current orthodoxy
maintains that biological processes of ontogenesis proceed differently
from the selectionist processes of phylogenesis, evolutionary
epistemologies would reflect this difference. For example, the theory of
``neural Darwinism'' as put forth by Edelman and Changeaux offers a
selectionist account of the ontogenetic development of the brain.
\end{quote}

\begin{quote}
\emph{avg\_grammar\_rating} : 5.0\\
\emph{avg\_answerability\_rating} : 3.7\\
\emph{sum\_yes\_meaningful} : 1\\
\emph{sum\_no\_meaning} : 2\\
\emph{sum\_maybe\_meaning} : 0
\end{quote}

\hypertarget{what-else-is-the-fitness-of-a-type-modified-as-a-function-of}{%
\subsection{What else is the fitness of a type modified as a function
of?}\label{what-else-is-the-fitness-of-a-type-modified-as-a-function-of}}

\begin{quote}
\emph{summarized\_paragraph} : Some relationships may be represented
without a matrix. Boyd and Richerson were interested in a special kind
of frequency dependent transmission bias in culture. \textbf{In such a
case, the operative fitness of the type is just the fitness as
calculated according to the usual factors, and then modified as a
function of the frequency of thetype.} The matrix is used to represent
relationships that are not directly related to each other, such as those
between two different types of data or between two types of information.
For example, the matrix can be used to show that a type of data is more
common than a type that is not common.
\end{quote}

\begin{quote}
\emph{avg\_grammar\_rating} : 3.3\\
\emph{avg\_answerability\_rating} : 3.7\\
\emph{sum\_yes\_meaningful} : 2\\
\emph{sum\_no\_meaning} : 1\\
\emph{sum\_maybe\_meaning} : 0
\end{quote}

\hypertarget{what-are-the-usual-factors-used-to-calculate-the-operative-fitness-of-a-type}{%
\subsection{What are the usual factors used to calculate the operative
fitness of a
type?}\label{what-are-the-usual-factors-used-to-calculate-the-operative-fitness-of-a-type}}

\begin{quote}
\emph{summarized\_paragraph} : Some relationships may be represented
without a matrix. Boyd and Richerson were interested in a special kind
of frequency dependent transmission bias in culture. \textbf{In such a
case, the operative fitness of the type is just the fitness as
calculated according to the usual factors, and then modified as a
function of the frequency of thetype.} The matrix is used to represent
relationships that are not directly related to each other, such as those
between two different types of data or between two types of information.
For example, the matrix can be used to show that a type of data is more
common than a type that is not common.
\end{quote}

\begin{quote}
\emph{avg\_grammar\_rating} : 4.7\\
\emph{avg\_answerability\_rating} : 2.3\\
\emph{sum\_yes\_meaningful} : 3\\
\emph{sum\_no\_meaning} : 0\\
\emph{sum\_maybe\_meaning} : 0
\end{quote}

\hypertarget{what-type-of-entities-are-the-main-difficulty-with-the-npo-model}{%
\subsection{What type of entities are the main difficulty with the NPO
model?}\label{what-type-of-entities-are-the-main-difficulty-with-the-npo-model}}

\begin{quote}
\emph{summarized\_paragraph} : Richard Dawkins' invention of the ``meme'
is the most popular attempt to understand cultural evolution.
\textbf{The main difficulty with this approach has been the problem of
how to provide specifications for the basic entities.} There appears to
be no such fundamental ``alphabet'' for the items of cultural
transmission. The project of `memetics' as a contending basis for
evolutionary epistemology is on hold pending an adequate understanding
of its basic ontology. The online Journal of Memetics contains some
early papers on memetics.
\end{quote}

\begin{quote}
\emph{avg\_grammar\_rating} : 4.3\\
\emph{avg\_answerability\_rating} : 3.3\\
\emph{sum\_yes\_meaningful} : 2\\
\emph{sum\_no\_meaning} : 1\\
\emph{sum\_maybe\_meaning} : 0
\end{quote}

\hypertarget{what-is-the-simplest-type-of-model}{%
\subsection{What is the simplest type of
model?}\label{what-is-the-simplest-type-of-model}}

\begin{quote}
\emph{summarized\_paragraph} : In the simplest sort of model, an
organism has to deal with an environment that has two states, ~and , and
has two possible responses. We suppose that what the organism does in
each state makes a difference to its fitness. Fitnesses are usually
written characterized by a matrix . \textbf{In the simplest of models,
we suppose that an organism's fitness is determined by its response to
an environment in a certain way.} We say that the organism's response to
the environment is affected by what it does in that state.
\end{quote}

\begin{quote}
\emph{avg\_grammar\_rating} : 5.0\\
\emph{avg\_answerability\_rating} : 4.0\\
\emph{sum\_yes\_meaningful} : 2\\
\emph{sum\_no\_meaning} : 1\\
\emph{sum\_maybe\_meaning} : 0
\end{quote}

\hypertarget{in-the-simplest-model-what-is-the-organisms-fitness-determined-by}{%
\subsection{In the simplest model , what is the organism's fitness
determined
by?}\label{in-the-simplest-model-what-is-the-organisms-fitness-determined-by}}

\begin{quote}
\emph{summarized\_paragraph} : In the simplest sort of model, an
organism has to deal with an environment that has two states, ~and , and
has two possible responses. We suppose that what the organism does in
each state makes a difference to its fitness. Fitnesses are usually
written characterized by a matrix . \textbf{In the simplest of models,
we suppose that an organism's fitness is determined by its response to
an environment in a certain way.} We say that the organism's response to
the environment is affected by what it does in that state.
\end{quote}

\begin{quote}
\emph{avg\_grammar\_rating} : 5.0\\
\emph{avg\_answerability\_rating} : 5.0\\
\emph{sum\_yes\_meaningful} : 2\\
\emph{sum\_no\_meaning} : 0\\
\emph{sum\_maybe\_meaning} : 1
\end{quote}

\hypertarget{what-is-the-most-likely-course-of-evolution}{%
\subsection{What is the most likely course of
evolution?}\label{what-is-the-most-likely-course-of-evolution}}

\begin{quote}
\emph{summarized\_paragraph} : Simulation results showed that virtually
all initial population distributions become dominated by one or the
other of the two signaling system strategies. The situation becomes more
complex when more realistic payoffs are introduced, for instance, that
the sender incurs a cost rather than automatically sharing the benefit
that the receiver gets from correct behavior for the environment.
\textbf{Even in such situations, however, the most likely course of
evolution is domination by a signaling system, the study found.} The
study was published in the journal Proceedings of the National Academy
of Sciences.
\end{quote}

\begin{quote}
\emph{avg\_grammar\_rating} : 5.0\\
\emph{avg\_answerability\_rating} : 5.0\\
\emph{sum\_yes\_meaningful} : 3\\
\emph{sum\_no\_meaning} : 0\\
\emph{sum\_maybe\_meaning} : 0
\end{quote}

\hypertarget{what-does-the-growth-of-knowledge-follow-in-biology}{%
\subsection{What does the growth of knowledge follow in
biology?}\label{what-does-the-growth-of-knowledge-follow-in-biology}}

\begin{quote}
\emph{summarized\_paragraph} : The Darwinian revolution of the
nineteenth century suggested an alternative approach first explored by
Dewey and the pragmatists. On this view, there is no sharp division of
labor between science and epistemology. Such approaches, in general, are
called naturalistic epistemologies, whether they are directly motivated
by evolutionary considerations or not. \textbf{Those which argue that
the growth of knowledge follows the pattern of evolution in biology are
called ``evolutionary~epistemologies.'' The results of evolutionary
biology and psychology are not ruled a priori irrelevant to the solution
of epistemological problems.}
\end{quote}

\begin{quote}
\emph{avg\_grammar\_rating} : 5.0\\
\emph{avg\_answerability\_rating} : 4.7\\
\emph{sum\_yes\_meaningful} : 3\\
\emph{sum\_no\_meaning} : 0\\
\emph{sum\_maybe\_meaning} : 0
\end{quote}

\hypertarget{what-are-the-results-of-evolutionary-biology-and-psychology-not-ruled-a-priori-to-the-solution-of-epistemological-problems}{%
\subsection{What are the results of evolutionary biology and psychology
not ruled a priori to the solution of epistemological
problems?}\label{what-are-the-results-of-evolutionary-biology-and-psychology-not-ruled-a-priori-to-the-solution-of-epistemological-problems}}

\begin{quote}
\emph{summarized\_paragraph} : The Darwinian revolution of the
nineteenth century suggested an alternative approach first explored by
Dewey and the pragmatists. On this view, there is no sharp division of
labor between science and epistemology. Such approaches, in general, are
called naturalistic epistemologies, whether they are directly motivated
by evolutionary considerations or not. \textbf{Those which argue that
the growth of knowledge follows the pattern of evolution in biology are
called ``evolutionary~epistemologies.'' The results of evolutionary
biology and psychology are not ruled a priori irrelevant to the solution
of epistemological problems.}
\end{quote}

\begin{quote}
\emph{avg\_grammar\_rating} : 5.0\\
\emph{avg\_answerability\_rating} : 3.3\\
\emph{sum\_yes\_meaningful} : 3\\
\emph{sum\_no\_meaning} : 0\\
\emph{sum\_maybe\_meaning} : 0
\end{quote}

\hypertarget{what-is-the-epidemiological-approach-to-the-study-of-cultural-transmission}{%
\subsection{What is the epidemiological approach to the study of
cultural
transmission?}\label{what-is-the-epidemiological-approach-to-the-study-of-cultural-transmission}}

\begin{quote}
\emph{summarized\_paragraph} : Evolutionary game theory models are
claimed to cover both processes in which strategies are inherited and
those in which they are imitated. This application is possible in the
absence of any specification of the underlying nature of strategies.
\textbf{This is sometimes referred to as the `epidemiological approach'
to the study of cultural transmission.} The comparison to infection is
due to the quantitative tools used in analysis rather than to any
presupposition regarding the underlying ontology of cultural transmitted
strategies. Population models have been used to good effect in modeling
cultural transmission processes.
\end{quote}

\begin{quote}
\emph{avg\_grammar\_rating} : 5.0\\
\emph{avg\_answerability\_rating} : 4.0\\
\emph{sum\_yes\_meaningful} : 3\\
\emph{sum\_no\_meaning} : 0\\
\emph{sum\_maybe\_meaning} : 0
\end{quote}

\hypertarget{what-type-of-transmission-is-studied-by-the-epidemiological-approach}{%
\subsection{What type of transmission is studied by the epidemiological
approach?}\label{what-type-of-transmission-is-studied-by-the-epidemiological-approach}}

\begin{quote}
\emph{summarized\_paragraph} : Evolutionary game theory models are
claimed to cover both processes in which strategies are inherited and
those in which they are imitated. This application is possible in the
absence of any specification of the underlying nature of strategies.
\textbf{This is sometimes referred to as the `epidemiological approach'
to the study of cultural transmission.} The comparison to infection is
due to the quantitative tools used in analysis rather than to any
presupposition regarding the underlying ontology of cultural transmitted
strategies. Population models have been used to good effect in modeling
cultural transmission processes.
\end{quote}

\begin{quote}
\emph{avg\_grammar\_rating} : 5.0\\
\emph{avg\_answerability\_rating} : 5.0\\
\emph{sum\_yes\_meaningful} : 3\\
\emph{sum\_no\_meaning} : 0\\
\emph{sum\_maybe\_meaning} : 0
\end{quote}

\hypertarget{what-is-the-strategy-used-in-the-experiment}{%
\subsection{What is the strategy used in the
experiment?}\label{what-is-the-strategy-used-in-the-experiment}}

\begin{quote}
\emph{summarized\_paragraph} : matrix-driven evolution can exhibit quite
complex behavior. More complex situations can be modeled, of course, but
additive matrices are the standard. For instance, chaotic behavior is
possible with as few as four strategies. where ~is type 's fitness in
situation . \textbf{where is type of situation and is the type of
strategy used in the experiment.. For more information, see:
http://www.cnn.com/2013/01/30/science/features/features-matrix-driven-evolution-and-chaos.html.}
\end{quote}

\begin{quote}
\emph{avg\_grammar\_rating} : 5.0\\
\emph{avg\_answerability\_rating} : 2.3\\
\emph{sum\_yes\_meaningful} : 1\\
\emph{sum\_no\_meaning} : 1\\
\emph{sum\_maybe\_meaning} : 1
\end{quote}

\hypertarget{what-is-the-difficulty-in-understanding-cognitive-behavior-as-a-product-of-evolution}{%
\subsection{What is the difficulty in understanding cognitive behavior
as a product of
evolution?}\label{what-is-the-difficulty-in-understanding-cognitive-behavior-as-a-product-of-evolution}}

\begin{quote}
\emph{summarized\_paragraph} : \textbf{The difficulty in understanding
cognitive behavior as the product of evolution is that there are at
least three very different evolutionary processes involved.} There is
the biological evolution of cognitive and perceptual mechanisms via
genetic inheritance. Second, there is the cultural evolution of
languages and concepts. Third, there's the trial-and-error learning
process that occurs during an individual's lifetime. understanding human
knowledge fully will require understanding the interaction between these
processes. This requires that we be able to model both processes of
biological and cultural evolution.
\end{quote}

\begin{quote}
\emph{avg\_grammar\_rating} : 5.0\\
\emph{avg\_answerability\_rating} : 5.0\\
\emph{sum\_yes\_meaningful} : 3\\
\emph{sum\_no\_meaning} : 0\\
\emph{sum\_maybe\_meaning} : 0
\end{quote}

\hypertarget{what-process-is-cognitive-behavior-a-product-of}{%
\subsection{What process is cognitive behavior a product
of?}\label{what-process-is-cognitive-behavior-a-product-of}}

\begin{quote}
\emph{summarized\_paragraph} : \textbf{The difficulty in understanding
cognitive behavior as the product of evolution is that there are at
least three very different evolutionary processes involved.} There is
the biological evolution of cognitive and perceptual mechanisms via
genetic inheritance. Second, there is the cultural evolution of
languages and concepts. Third, there's the trial-and-error learning
process that occurs during an individual's lifetime. understanding human
knowledge fully will require understanding the interaction between these
processes. This requires that we be able to model both processes of
biological and cultural evolution.
\end{quote}

\begin{quote}
\emph{avg\_grammar\_rating} : 4.0\\
\emph{avg\_answerability\_rating} : 4.0\\
\emph{sum\_yes\_meaningful} : 3\\
\emph{sum\_no\_meaning} : 0\\
\emph{sum\_maybe\_meaning} : 0
\end{quote}

\hypertarget{the-brain-of-what-animal-is-different}{%
\subsection{The brain of what animal is
different?}\label{the-brain-of-what-animal-is-different}}

\begin{quote}
\emph{summarized\_paragraph} : There are at least two different
approaches that have been taken to modeling multi-level evolution.
\textbf{There are also two different ways of looking at the evolution of
the human brain.} The most common approach is to look at the development
of the brain from the inside out. The other is to take a look at how the
brain has evolved from the outside in to the inside. The best way to
model evolution is by looking at how it has evolved in the past. The
most popular way is to study the history of evolution from the top down
to the bottom.
\end{quote}

\begin{quote}
\emph{avg\_grammar\_rating} : 3.3\\
\emph{avg\_answerability\_rating} : 2.7\\
\emph{sum\_yes\_meaningful} : 1\\
\emph{sum\_no\_meaning} : 2\\
\emph{sum\_maybe\_meaning} : 0
\end{quote}

\hypertarget{the-brain-of-what-animal-is-different-1}{%
\subsection{The brain of what animal is
different?}\label{the-brain-of-what-animal-is-different-1}}

\begin{quote}
\emph{summarized\_paragraph} : There are at least two different
approaches that have been taken to modeling multi-level evolution.
\textbf{There are also two different ways of looking at the evolution of
the human brain.} The most common approach is to look at the development
of the brain from the inside out. The other is to take a look at how the
brain has evolved from the outside in to the inside. The best way to
model evolution is by looking at how it has evolved in the past. The
most popular way is to study the history of evolution from the top down
to the bottom.
\end{quote}

\begin{quote}
\emph{avg\_grammar\_rating} : 3.0\\
\emph{avg\_answerability\_rating} : 3.3\\
\emph{sum\_yes\_meaningful} : 1\\
\emph{sum\_no\_meaning} : 2\\
\emph{sum\_maybe\_meaning} : 0
\end{quote}

\hypertarget{what-part-of-the-body-is-the-most-often-studied}{%
\subsection{What part of the body is the most often
studied?}\label{what-part-of-the-body-is-the-most-often-studied}}

\begin{quote}
\emph{summarized\_paragraph} : There are at least two different
approaches that have been taken to modeling multi-level evolution. There
are also two different ways of looking at the evolution of the human
brain. \textbf{The most common approach is to look at the development of
the brain from the inside out.} The other is to take a look at how the
brain has evolved from the outside in to the inside. The best way to
model evolution is by looking at how it has evolved in the past. The
most popular way is to study the history of evolution from the top down
to the bottom.
\end{quote}

\begin{quote}
\emph{avg\_grammar\_rating} : 5.0\\
\emph{avg\_answerability\_rating} : 4.7\\
\emph{sum\_yes\_meaningful} : 3\\
\emph{sum\_no\_meaning} : 0\\
\emph{sum\_maybe\_meaning} : 0
\end{quote}

\hypertarget{how-many-strategies-are-there}{%
\subsection{How many strategies are
there?}\label{how-many-strategies-are-there}}

\begin{quote}
\emph{summarized\_paragraph} : Since players will be both sender and
receiver, they must have a strategy for each situation. \textbf{There
are sixteen such strategies, and we suppose them to be either inherited
from biological parents, or imitated on the basis of perceived success
in terms of points earned.} Strategies ~and ~are signaling systems, in
that if both players play the same one of these two strategies they will
always get their payoff. All of the other strategies involve , or ,
which results in the same act being performed no matter what the
external state is.
\end{quote}

\begin{quote}
\emph{avg\_grammar\_rating} : 5.0\\
\emph{avg\_answerability\_rating} : 4.3\\
\emph{sum\_yes\_meaningful} : 1\\
\emph{sum\_no\_meaning} : 2\\
\emph{sum\_maybe\_meaning} : 0
\end{quote}

\hypertarget{what-are-there-sixteen-such-as}{%
\subsection{What are there sixteen such
as?}\label{what-are-there-sixteen-such-as}}

\begin{quote}
\emph{summarized\_paragraph} : Since players will be both sender and
receiver, they must have a strategy for each situation. \textbf{There
are sixteen such strategies, and we suppose them to be either inherited
from biological parents, or imitated on the basis of perceived success
in terms of points earned.} Strategies ~and ~are signaling systems, in
that if both players play the same one of these two strategies they will
always get their payoff. All of the other strategies involve , or ,
which results in the same act being performed no matter what the
external state is.
\end{quote}

\begin{quote}
\emph{avg\_grammar\_rating} : 2.3\\
\emph{avg\_answerability\_rating} : 2.0\\
\emph{sum\_yes\_meaningful} : 0\\
\emph{sum\_no\_meaning} : 3\\
\emph{sum\_maybe\_meaning} : 0
\end{quote}

\hypertarget{what-kind-of-model-is-used-to-describe-the-evolution-of-flexible-responses}{%
\subsection{What kind of model is used to describe the evolution of
flexible
responses?}\label{what-kind-of-model-is-used-to-describe-the-evolution-of-flexible-responses}}

\begin{quote}
\emph{summarized\_paragraph} : This simple model demonstrates that
whether or not flexible responses are adaptive depends on the particular
characteristics of the fitness differences that the responses make. The
particular result is calculated assuming that detecting the
environmental state and the flexible response system is free in
evolutionary terms. More complete analyses would include the costs of
these factors, such as the probability of the various states of the
environment, and the reliability of the detector. \textbf{The results
are based on a simple model of the evolution of flexible responses in
the animal kingdom, using a variety of models.}
\end{quote}

\begin{quote}
\emph{avg\_grammar\_rating} : 5.0\\
\emph{avg\_answerability\_rating} : 2.3\\
\emph{sum\_yes\_meaningful} : 2\\
\emph{sum\_no\_meaning} : 0\\
\emph{sum\_maybe\_meaning} : 1
\end{quote}

\hypertarget{what-is-used-to-describe-the-evolution-of-flexible-responses}{%
\subsection{What is used to describe the evolution of flexible
responses?}\label{what-is-used-to-describe-the-evolution-of-flexible-responses}}

\begin{quote}
\emph{summarized\_paragraph} : This simple model demonstrates that
whether or not flexible responses are adaptive depends on the particular
characteristics of the fitness differences that the responses make. The
particular result is calculated assuming that detecting the
environmental state and the flexible response system is free in
evolutionary terms. More complete analyses would include the costs of
these factors, such as the probability of the various states of the
environment, and the reliability of the detector. \textbf{The results
are based on a simple model of the evolution of flexible responses in
the animal kingdom, using a variety of models.}
\end{quote}

\begin{quote}
\emph{avg\_grammar\_rating} : 5.0\\
\emph{avg\_answerability\_rating} : 3.0\\
\emph{sum\_yes\_meaningful} : 2\\
\emph{sum\_no\_meaning} : 1\\
\emph{sum\_maybe\_meaning} : 0
\end{quote}

\hypertarget{from-whom-did-this-project-get-its-modern-stamp}{%
\subsection{From whom did this project get its modern
stamp?}\label{from-whom-did-this-project-get-its-modern-stamp}}

\begin{quote}
\emph{summarized\_paragraph} : Traditional epistemology has its roots in
Plato and the ancient skeptics. The bonds that hold the reconstruction
of human knowledge together are the justificational and evidential
relations which enable us to distinguish knowledge from true belief.
\textbf{This project got its modern stamp from Descartes and comes in
empiricist as well as rationalist versions which in turn can be given
either a foundational or coherentist twist.} The two strands are woven
together by a common theme. The bond that holds the Reconstruction of
Human knowledge together is the bonds that allow us to distinguishing
knowledge fromtrue belief.
\end{quote}

\begin{quote}
\emph{avg\_grammar\_rating} : 5.0\\
\emph{avg\_answerability\_rating} : 5.0\\
\emph{sum\_yes\_meaningful} : 1\\
\emph{sum\_no\_meaning} : 1\\
\emph{sum\_maybe\_meaning} : 1
\end{quote}

\hypertarget{infinite-regress-arguments}{%
\section{\texorpdfstring{\href{https://plato.stanford.edu/entries/infinite-regress/index.html}{Infinite
Regress
Arguments}}{Infinite Regress Arguments}}\label{infinite-regress-arguments}}

\hypertarget{according-to-the-defender-of-the-a---series-how-does-reality-change}{%
\subsection{According to the defender of the a - series , how does
reality
change?}\label{according-to-the-defender-of-the-a---series-how-does-reality-change}}

\begin{quote}
\emph{summarized\_paragraph} : The way things were and the way things
will be seems to be part of reality in a way that, for example, Bilbo's
finding the One Ring is not. How can they both contribute to the way
reality is if they are incompatible? \textbf{The defender of the
A-series replies that in giving the complete account of how reality is
we have to take seriously the fact that reality changes and that it is,
therefore, different ways successively, and there is no inconsistency in
things being one way and then another.}
\end{quote}

\begin{quote}
\emph{avg\_grammar\_rating} : nan\\
\emph{avg\_answerability\_rating} : nan\\
\emph{sum\_yes\_meaningful} : 0\\
\emph{sum\_no\_meaning} : 0\\
\emph{sum\_maybe\_meaning} : 0
\end{quote}

\hypertarget{what-type-of-commitments-lead-to-a-regress-that-is-objectionable-to-both-theorists-and-one-other}{%
\subsection{What type of commitments lead to a regress that is
objectionable to both theorists and one
other?}\label{what-type-of-commitments-lead-to-a-regress-that-is-objectionable-to-both-theorists-and-one-other}}

\begin{quote}
\emph{summarized\_paragraph} : In the previous section we saw two
theories generating similar regresses, but where one is found
unobjectionable whereas the other is found objectionable due to the
different things we think we know. \textbf{We could also have cases
where a single theory yields a regress that is objectionable by the
lights of one theorist and not another, as a result of their differing
theoretical commitments leading one but not the other to think that a
feature revealed by the regress is a vice.} In the next section we will
look at the different arguments that can be made for and against
different theories.
\end{quote}

\begin{quote}
\emph{avg\_grammar\_rating} : nan\\
\emph{avg\_answerability\_rating} : nan\\
\emph{sum\_yes\_meaningful} : 0\\
\emph{sum\_no\_meaning} : 0\\
\emph{sum\_maybe\_meaning} : 0
\end{quote}

\hypertarget{how-much-deferred-would-a-dependent-entity-be-in-the-case-of-an-infinite-regress-of-ontological-dependence}{%
\subsection{How much deferred would a dependent entity be in the case of
an infinite regress of ontological
dependence?}\label{how-much-deferred-would-a-dependent-entity-be-in-the-case-of-an-infinite-regress-of-ontological-dependence}}

\begin{quote}
\emph{summarized\_paragraph} : \textbf{Schaffer claims that in the case
of an infinite regress of ontological dependence, with each entity
depending on the next in the chain, and no independent entities, being
would be `infinitely deferred, never achieved' The idea seems to be that
a dependent entity only has the being it has on condition of something
else having being.} If this process never stops, the promissory note is
never paid, in which case, allegedly, the existence of all these things
could never get off the ground in the first place.
\end{quote}

\begin{quote}
\emph{avg\_grammar\_rating} : 5.0\\
\emph{avg\_answerability\_rating} : 5.0\\
\emph{sum\_yes\_meaningful} : 1\\
\emph{sum\_no\_meaning} : 0\\
\emph{sum\_maybe\_meaning} : 0
\end{quote}

\hypertarget{what-type-of-person-is-to-hold-that-each-belief-is-justified-in-virtue-of-the-next-one-being-justified}{%
\subsection{What type of person is to hold that each belief is justified
in virtue of the next one being
justified?}\label{what-type-of-person-is-to-hold-that-each-belief-is-justified-in-virtue-of-the-next-one-being-justified}}

\begin{quote}
\emph{summarized\_paragraph} : Epistemic Infinitists embrace the
infinite regress of reasons and argue that it is not vicious.
\textbf{One response for the Infinitist to make to the regress argument
is to hold that each belief is justified in virtue of the next one being
justified, but to claim that this is not a problem.} This is not what
the sequence was meant to explain. All that we need is an explanation
for each belief concerning why it is justified, and this we have.
Another example.
\end{quote}

\begin{quote}
\emph{avg\_grammar\_rating} : nan\\
\emph{avg\_answerability\_rating} : nan\\
\emph{sum\_yes\_meaningful} : 0\\
\emph{sum\_no\_meaning} : 0\\
\emph{sum\_maybe\_meaning} : 0
\end{quote}

\hypertarget{what-does-one-see-in-the-way-the-commitment-to-reality-is-each-of-two-incompatible-ways}{%
\subsection{What does one see in the way the commitment to reality is
each of two incompatible
ways?}\label{what-does-one-see-in-the-way-the-commitment-to-reality-is-each-of-two-incompatible-ways}}

\begin{quote}
\emph{summarized\_paragraph} : McTaggart's regress looks vicious or
benign depending on whether one is content to grant the legitimacy of
the notion of temporal succession. There is never, at any stage, a
contradiction, for we are never forced to say that a thing has
incompatible properties. \textbf{If, by contrast, one sees in it simply
an attempt to paper over what is ultimately a contradiction inherent in
the commitment to reality being each of two incompatible ways, the way
it supposedly is now and the way It supposedly was or will be.}
\end{quote}

\begin{quote}
\emph{avg\_grammar\_rating} : nan\\
\emph{avg\_answerability\_rating} : nan\\
\emph{sum\_yes\_meaningful} : 0\\
\emph{sum\_no\_meaning} : 0\\
\emph{sum\_maybe\_meaning} : 0
\end{quote}

\hypertarget{the-way-it-is-now-and-the-way-it-is-are-two-ways-that-reality-is-different-from-reality}{%
\subsection{The way it is now and the way it is are two ways that
reality is different from
reality?}\label{the-way-it-is-now-and-the-way-it-is-are-two-ways-that-reality-is-different-from-reality}}

\begin{quote}
\emph{summarized\_paragraph} : McTaggart's regress looks vicious or
benign depending on whether one is content to grant the legitimacy of
the notion of temporal succession. There is never, at any stage, a
contradiction, for we are never forced to say that a thing has
incompatible properties. \textbf{If, by contrast, one sees in it simply
an attempt to paper over what is ultimately a contradiction inherent in
the commitment to reality being each of two incompatible ways, the way
it supposedly is now and the way It supposedly was or will be.}
\end{quote}

\begin{quote}
\emph{avg\_grammar\_rating} : nan\\
\emph{avg\_answerability\_rating} : nan\\
\emph{sum\_yes\_meaningful} : 0\\
\emph{sum\_no\_meaning} : 0\\
\emph{sum\_maybe\_meaning} : 0
\end{quote}

\hypertarget{what-is-the-infinite-thing-that-there-is-no-impossibility-in}{%
\subsection{What is the infinite thing that there is no impossibility
in?}\label{what-is-the-infinite-thing-that-there-is-no-impossibility-in}}

\begin{quote}
\emph{summarized\_paragraph} : Ross Cameron applies considerations of
theoretical parsimony to the case of infinite chains of ontological
dependence. \textbf{While allowing that there is no impossibility in an
infinite regress of things, each ontologically dependent on the next,
Cameron argues that we can still have reason to reject such a theory on
parsimony considerations.} For example, a physical theory that
postulates one unified force to explain all phenomena would, other
things being equal, be preferable to one that postsulates four
fundamental forces. (Orilia argues, contra Cameron, thatthere is no
unified explanation provided by Metaphysical Foundationalism over
theories with infinite ontological descent.)
\end{quote}

\begin{quote}
\emph{avg\_grammar\_rating} : nan\\
\emph{avg\_answerability\_rating} : nan\\
\emph{sum\_yes\_meaningful} : 0\\
\emph{sum\_no\_meaning} : 0\\
\emph{sum\_maybe\_meaning} : 0
\end{quote}

\hypertarget{what-does-each-of-the-infinitely-many-s-have}{%
\subsection{What does each of the infinitely many ~s
have?}\label{what-does-each-of-the-infinitely-many-s-have}}

\begin{quote}
\emph{summarized\_paragraph} : Some philosophers have argued that when
we have an infinite regress, with the -ness of each ~being accounted for
by appeal to another ~that is , then we do indeed lack a global
explanation of why there are things that are . But we nevertheless have
a local explanation for each of the infinitely many \s as to why it is .
Distinguish between aLocal explanation of the of some particular ~and a
global explanations of any things at all.
\end{quote}

\begin{quote}
\emph{avg\_grammar\_rating} : nan\\
\emph{avg\_answerability\_rating} : nan\\
\emph{sum\_yes\_meaningful} : 0\\
\emph{sum\_no\_meaning} : 0\\
\emph{sum\_maybe\_meaning} : 0
\end{quote}

\hypertarget{what-does-it-take-for-the-regress-to-do}{%
\subsection{What does it take for the regress to
do?}\label{what-does-it-take-for-the-regress-to-do}}

\begin{quote}
\emph{summarized\_paragraph} : Bliss rejects the idea that in an
infinitely descending chain of ontological dependence, being would never
be achieved. Having a property dependent on some condition is
nevertheless to have that property, so there is no pressure to conclude
that nothing in the infinite series would exist. Rather, they all exist,
and the existence of each is perfectly well accounted for: it exists
because the next thing in the sequence does. \textbf{Why there is an
infinite chain of existing entities at all is not accounted for, but
Bliss says it is a mistake to think that the regress was ever supposed
to account for that.}
\end{quote}

\begin{quote}
\emph{avg\_grammar\_rating} : nan\\
\emph{avg\_answerability\_rating} : nan\\
\emph{sum\_yes\_meaningful} : 0\\
\emph{sum\_no\_meaning} : 0\\
\emph{sum\_maybe\_meaning} : 0
\end{quote}

\hypertarget{who-says-it-is-a-mistake-to-think-that-the-regress-was-ever-supposed-to-account-for-that}{%
\subsection{Who says it is a mistake to think that the regress was ever
supposed to account for
that?}\label{who-says-it-is-a-mistake-to-think-that-the-regress-was-ever-supposed-to-account-for-that}}

\begin{quote}
\emph{summarized\_paragraph} : Bliss rejects the idea that in an
infinitely descending chain of ontological dependence, being would never
be achieved. Having a property dependent on some condition is
nevertheless to have that property, so there is no pressure to conclude
that nothing in the infinite series would exist. Rather, they all exist,
and the existence of each is perfectly well accounted for: it exists
because the next thing in the sequence does. \textbf{Why there is an
infinite chain of existing entities at all is not accounted for, but
Bliss says it is a mistake to think that the regress was ever supposed
to account for that.}
\end{quote}

\begin{quote}
\emph{avg\_grammar\_rating} : nan\\
\emph{avg\_answerability\_rating} : nan\\
\emph{sum\_yes\_meaningful} : 0\\
\emph{sum\_no\_meaning} : 0\\
\emph{sum\_maybe\_meaning} : 0
\end{quote}

\hypertarget{what-does-the-same-principles-of-a-theory-generate-that-lead-to-a-contradiction}{%
\subsection{What does the same principles of a theory generate that lead
to a
contradiction?}\label{what-does-the-same-principles-of-a-theory-generate-that-lead-to-a-contradiction}}

\begin{quote}
\emph{summarized\_paragraph} : \textbf{One such kind of case is when the
very same principles of a theory that generate the regress also lead to
a contradiction.} If this is so then it does not matter what we think
about infinite regress in general, we will of course have reason to
reject the theory, because it is contradictory. Two such examples are
discussed by Daniel Nolan , and we will recount one of them here. The
theory in question is Plato's theory of Forms, and the regress objection
is Parmenides's Third Man objection.
\end{quote}

\begin{quote}
\emph{avg\_grammar\_rating} : nan\\
\emph{avg\_answerability\_rating} : nan\\
\emph{sum\_yes\_meaningful} : 0\\
\emph{sum\_no\_meaning} : 0\\
\emph{sum\_maybe\_meaning} : 0
\end{quote}

\hypertarget{what-is-an-infinite-sequence-of}{%
\subsection{What is an infinite sequence
of?}\label{what-is-an-infinite-sequence-of}}

\begin{quote}
\emph{summarized\_paragraph} : Klein: It is overwhelmingly plausible
that ~can only serve as the ontological ground of ~because ~itself
exists, or is the way it is. But it is not forced on us to hold that can
only be a reason for because ~is itself justified, and this is why
Klein's response to the epistemic regress is available. It need be no
part of the explanation for why ~is a reason that ~itself be justified.
\textbf{So while there is indeed an infinite sequence of propositions,
each of which is a. reason for the previous one on the list, at no stage
is the fact that one proposition is a reasons for another hostage to any
other.}
\end{quote}

\begin{quote}
\emph{avg\_grammar\_rating} : nan\\
\emph{avg\_answerability\_rating} : nan\\
\emph{sum\_yes\_meaningful} : 0\\
\emph{sum\_no\_meaning} : 0\\
\emph{sum\_maybe\_meaning} : 0
\end{quote}

\hypertarget{what-is-the-regress}{%
\subsection{What is the regress?}\label{what-is-the-regress}}

\begin{quote}
\emph{summarized\_paragraph} : If we are seeking an explanation of how
or why a thing exists, and so on ad infinitum, the regress is benign. In
order to explain facts about my existence, we can make recourse to the
existence of my parents, my vital organs, etc. At each stage of the
regress, we have a satisfactory explanation of that for which we
areseeking one. \textbf{The regress is harmless because where the
explanans and explanandum are not of the same form, we don't need to
regress.}
\end{quote}

\begin{quote}
\emph{avg\_grammar\_rating} : nan\\
\emph{avg\_answerability\_rating} : nan\\
\emph{sum\_yes\_meaningful} : 0\\
\emph{sum\_no\_meaning} : 0\\
\emph{sum\_maybe\_meaning} : 0
\end{quote}

\hypertarget{who-concludes-that-we-end-up-attributing-each-a---property-to-every-time}{%
\subsection{Who concludes that we end up attributing each a - property
to every
time?}\label{who-concludes-that-we-end-up-attributing-each-a---property-to-every-time}}

\begin{quote}
\emph{summarized\_paragraph} : The A-series of time is the sequence of
times one of which is present, others past, and others future. McTaggart
concludes that we end up attributing each A-property to every time (and
therefore to every event in time) This is absurd, because the
A-properties are incompatible: to have one is to have neither of the
others. Each time both has only one such property, and all such
properties. The A- series cannot be real, says Mc taggart.
\end{quote}

\begin{quote}
\emph{avg\_grammar\_rating} : nan\\
\emph{avg\_answerability\_rating} : nan\\
\emph{sum\_yes\_meaningful} : 0\\
\emph{sum\_no\_meaning} : 0\\
\emph{sum\_maybe\_meaning} : 0
\end{quote}

\hypertarget{blbos-doing-what-is-not-part-of-reality}{%
\subsection{Blbo's doing what is not part of
reality?}\label{blbos-doing-what-is-not-part-of-reality}}

\begin{quote}
\emph{summarized\_paragraph} : \textbf{The way things were and the way
things will be seems to be part of reality in a way that, for example,
Bilbo's finding the One Ring is not.} How can they both contribute to
the way reality is if they are incompatible? The defender of the
A-series replies that in giving the complete account of how reality is
we have to take seriously the fact that reality changes and that it is,
therefore, different ways successively, and there is no inconsistency in
things being one way and then another.
\end{quote}

\begin{quote}
\emph{avg\_grammar\_rating} : nan\\
\emph{avg\_answerability\_rating} : nan\\
\emph{sum\_yes\_meaningful} : 0\\
\emph{sum\_no\_meaning} : 0\\
\emph{sum\_maybe\_meaning} : 0
\end{quote}

\hypertarget{there-is-no-pressure-to-do-what}{%
\subsection{There is no pressure to do
what?}\label{there-is-no-pressure-to-do-what}}

\begin{quote}
\emph{summarized\_paragraph} : Bliss rejects the idea that in an
infinitely descending chain of ontological dependence, being would never
be achieved. \textbf{Having a property dependent on some condition is
nevertheless to have that property, so there is no pressure to conclude
that nothing in the infinite series would exist.} Rather, they all
exist, and the existence of each is perfectly well accounted for: it
exists because the next thing in the sequence does. Why there is an
infinite chain of existing entities at all is not accounted for, but
Bliss says it is a mistake to think that the regress was ever supposed
to account for that.
\end{quote}

\begin{quote}
\emph{avg\_grammar\_rating} : nan\\
\emph{avg\_answerability\_rating} : nan\\
\emph{sum\_yes\_meaningful} : 0\\
\emph{sum\_no\_meaning} : 0\\
\emph{sum\_maybe\_meaning} : 0
\end{quote}

\hypertarget{how-is-grounding-asymmetric}{%
\subsection{How is grounding
asymmetric?}\label{how-is-grounding-asymmetric}}

\begin{quote}
\emph{summarized\_paragraph} : \textbf{If we were providing the
metaphysical grounds of rates of change, Smart might be right that this
would lead to a vicious regress, since arguably grounding is
asymmetric.} But that is not what is going on. When we explain the speed
of the car by appeal to the passage of time, we're not providing the
ontological grounds of its speed. Likewise for the rate of time's
passage itself: we are not seeking to provide the ontology of time in
stating its rate. Rather, when we compare the two changes, we are simply
trying to illuminate one or both of those changes by pointing to the way
they relate.
\end{quote}

\begin{quote}
\emph{avg\_grammar\_rating} : nan\\
\emph{avg\_answerability\_rating} : nan\\
\emph{sum\_yes\_meaningful} : 0\\
\emph{sum\_no\_meaning} : 0\\
\emph{sum\_maybe\_meaning} : 0
\end{quote}

\hypertarget{how-is-grounding-asymmetric-1}{%
\subsection{How is grounding
asymmetric?}\label{how-is-grounding-asymmetric-1}}

\begin{quote}
\emph{summarized\_paragraph} : \textbf{If we were providing the
metaphysical grounds of rates of change, Smart might be right that this
would lead to a vicious regress, since arguably grounding is
asymmetric.} But that is not what is going on. When we explain the speed
of the car by appeal to the passage of time, we're not providing the
ontological grounds of its speed. Likewise for the rate of time's
passage itself: we are not seeking to provide the ontology of time in
stating its rate. Rather, when we compare the two changes, we are simply
trying to illuminate one or both of those changes by pointing to the way
they relate.
\end{quote}

\begin{quote}
\emph{avg\_grammar\_rating} : nan\\
\emph{avg\_answerability\_rating} : nan\\
\emph{sum\_yes\_meaningful} : 0\\
\emph{sum\_no\_meaning} : 0\\
\emph{sum\_maybe\_meaning} : 0
\end{quote}

\hypertarget{every-being-derives-its-reality-from-the-reality-of-what}{%
\subsection{Every being derives its reality from the reality of
what?}\label{every-being-derives-its-reality-from-the-reality-of-what}}

\begin{quote}
\emph{summarized\_paragraph} : Where there are only beings by
aggregation {[}composite objects{]}, there are no real beings. For every
being by aggregation presupposes beings endowed with real unity
{[}simples{]}. \textbf{Every being derives its reality only from the
reality of those beings of which it is composed, so that it will not
have any reality at all if it is not itself a being of aggregation.} We
must still seek further grounds for its reality, grounds which can never
be found in this way, if we must always continue to seek for them.
\end{quote}

\begin{quote}
\emph{avg\_grammar\_rating} : nan\\
\emph{avg\_answerability\_rating} : nan\\
\emph{sum\_yes\_meaningful} : 0\\
\emph{sum\_no\_meaning} : 0\\
\emph{sum\_maybe\_meaning} : 0
\end{quote}

\hypertarget{what-does-the-theory-of-infinite-regress-do-to-an-infinite-regress}{%
\subsection{What does the theory of infinite regress do to an infinite
regress?}\label{what-does-the-theory-of-infinite-regress-do-to-an-infinite-regress}}

\begin{quote}
\emph{summarized\_paragraph} : \textbf{This is an easy case, because we
don't have to adjudicate on whether the fact that the theory leads to an
infinite regress is itself objectionable.} The principles that lead to
regress also lead to contradiction, and we know that a theory's being
contradictory is a good reason to reject it, whether it leads to regress
or not. This is anEasy case, Because we don't have to. Theory leads to
infinite regress, but it also leads to contradiction.
\end{quote}

\begin{quote}
\emph{avg\_grammar\_rating} : nan\\
\emph{avg\_answerability\_rating} : nan\\
\emph{sum\_yes\_meaningful} : 0\\
\emph{sum\_no\_meaning} : 0\\
\emph{sum\_maybe\_meaning} : 0
\end{quote}

\hypertarget{what-does-the-theory-of-infinite-theory-lead-to}{%
\subsection{What does the theory of infinite theory lead
to?}\label{what-does-the-theory-of-infinite-theory-lead-to}}

\begin{quote}
\emph{summarized\_paragraph} : \textbf{This is an easy case, because we
don't have to adjudicate on whether the fact that the theory leads to an
infinite regress is itself objectionable.} The principles that lead to
regress also lead to contradiction, and we know that a theory's being
contradictory is a good reason to reject it, whether it leads to regress
or not. This is anEasy case, Because we don't have to. Theory leads to
infinite regress, but it also leads to contradiction.
\end{quote}

\begin{quote}
\emph{avg\_grammar\_rating} : nan\\
\emph{avg\_answerability\_rating} : nan\\
\emph{sum\_yes\_meaningful} : 0\\
\emph{sum\_no\_meaning} : 0\\
\emph{sum\_maybe\_meaning} : 0
\end{quote}

\hypertarget{what-is-an-infinite-regress-taken-to-reveal}{%
\subsection{What is an infinite regress taken to
reveal?}\label{what-is-an-infinite-regress-taken-to-reveal}}

\begin{quote}
\emph{summarized\_paragraph} : \textbf{In section 1 we looked at cases
where an infinite regress is taken to reveal some feature that might,
possibly depending on your other theoretical commitments, be taken to be
a reason to reject a theory.} But sometimes the regress itself is taken
by some to be an objectionable feature of the theory that yields it. In
this section we look at cases in which the regress is seen to reveal a
feature of a theory independent of its leading to regress that is a
reasons to reject it. We conclude by looking at a case in which a theory
leads to a regress that leads to an objection to it.
\end{quote}

\begin{quote}
\emph{avg\_grammar\_rating} : nan\\
\emph{avg\_answerability\_rating} : nan\\
\emph{sum\_yes\_meaningful} : 0\\
\emph{sum\_no\_meaning} : 0\\
\emph{sum\_maybe\_meaning} : 0
\end{quote}

\hypertarget{what-type-of-entity-ultimately-comes-from-the-fundamental-thing-at-the-bottom-of-the-chain}{%
\subsection{What type of entity ultimately comes from the fundamental
thing at the bottom of the
chain?}\label{what-type-of-entity-ultimately-comes-from-the-fundamental-thing-at-the-bottom-of-the-chain}}

\begin{quote}
\emph{summarized\_paragraph} : If the chain is infinite, the being of
any thing is, arguably, as mysterious as Anne's new bag of sugar. The
explanation of where it came from is always postponed, and its presence
in the system as a whole unexplained. \textbf{And for any finite chain,
no matter how long, we can say where theBeing of any dependent entity
ultimately comes from: from the fundamental thing at the bottom of the
chain.} So, at least, goes the regress objection. As with sugar,
likewise with being---or justification, or goodness, or whatever feature
we aim to account for.
\end{quote}

\begin{quote}
\emph{avg\_grammar\_rating} : nan\\
\emph{avg\_answerability\_rating} : nan\\
\emph{sum\_yes\_meaningful} : 0\\
\emph{sum\_no\_meaning} : 0\\
\emph{sum\_maybe\_meaning} : 0
\end{quote}

\hypertarget{markosians-maneuver-is-possible-because-in-giving-the-rate-of-one-process-of-change-by-what}{%
\subsection{Markosian's maneuver is possible because in giving the rate
of one process of change by
what?}\label{markosians-maneuver-is-possible-because-in-giving-the-rate-of-one-process-of-change-by-what}}

\begin{quote}
\emph{summarized\_paragraph} : \textbf{Markosian's maneuver is possible
because in giving the rate of one process of change by appeal to a
second process, we are not saying what makes it the case that the first
change occurs at the rate it does.} Not that an hour of time passes
while the car moves a distance of 40 miles, for that is merely a
re-description of the fact in question: a way of describing the rate. In
the case of time itself, defenders of the view that time passes may
plausibly claim that it is simply the nature of time.
\end{quote}

\begin{quote}
\emph{avg\_grammar\_rating} : nan\\
\emph{avg\_answerability\_rating} : nan\\
\emph{sum\_yes\_meaningful} : 0\\
\emph{sum\_no\_meaning} : 0\\
\emph{sum\_maybe\_meaning} : 0
\end{quote}

\hypertarget{what-does-it-seem-that-there-are-infinitely-descending-chains-of}{%
\subsection{What does it seem that there are infinitely descending
chains
of?}\label{what-does-it-seem-that-there-are-infinitely-descending-chains-of}}

\begin{quote}
\emph{summarized\_paragraph} : Bliss argues that it is not necessarily a
mark against infinitely descending chains of ontological dependence that
it leaves this global fact---why does anything have being in the first
place?---unexplained. She says that this is not a reason to deny the
existence of the world, but to ask why it has been created in the way
that it has. Bliss says that the answer is that the world was created in
a way that makes it impossible to know why it was created. She argues
that the question is not whether it is possible, but why it happened.
\end{quote}

\begin{quote}
\emph{avg\_grammar\_rating} : nan\\
\emph{avg\_answerability\_rating} : nan\\
\emph{sum\_yes\_meaningful} : 0\\
\emph{sum\_no\_meaning} : 0\\
\emph{sum\_maybe\_meaning} : 0
\end{quote}

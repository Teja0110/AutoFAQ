\hypertarget{the-epistemology-of-religion}{%
\section{\texorpdfstring{\href{https://plato.stanford.edu/entries/religion-epistemology/index.html}{The
Epistemology of
Religion}}{The Epistemology of Religion}}\label{the-epistemology-of-religion}}

\hypertarget{according-to-evidentialism-what-is-at-best-probable-for-a-god}{%
\subsection{According to evidentialism , what is at best probable for a
god?}\label{according-to-evidentialism-what-is-at-best-probable-for-a-god}}

\begin{quote}
\emph{summarized\_paragraph} : \textbf{According to evidentialism it
follows that if the arguments for there being a God, including any
arguments from religious experience, are at best probable ones, and if,
as most hold, God's existence is not self-evident then no one would be
justified in having full belief that there is a God.} And the same holds
for other religious beliefs. It would not be justified to believe even
partially (i.e., with less than full confidence) if there is not a
balance of evidence for belief.
\end{quote}

\begin{quote}
\emph{avg\_grammar\_rating} : nan\\
\emph{avg\_answerability\_rating} : nan\\
\emph{sum\_yes\_meaningful} : 0\\
\emph{sum\_no\_meaning} : 0\\
\emph{sum\_maybe\_meaning} : 0
\end{quote}

\hypertarget{what-is-an-example-of-an-epistemological-theory-that-conflicts-with-the-holding-of-religious-beliefs}{%
\subsection{What is an example of an epistemological theory that
conflicts with the holding of religious
beliefs?}\label{what-is-an-example-of-an-epistemological-theory-that-conflicts-with-the-holding-of-religious-beliefs}}

\begin{quote}
\emph{summarized\_paragraph} : \textbf{Clifford: If an epistemological
theory such as evidentialism conflicts with the holding of religious
beliefs then that is so much the worse for the theory.} At the other
extreme from Clifford is the position of fideism, namely, that if a
theory conflicts with a religious belief then it is bad for that theory.
Clifford: If a theory like evidentialist conflicts with religious
beliefs, then it's bad for the epistemology theory. If it doesn't
conflict with religious belief, it's good for it.
\end{quote}

\begin{quote}
\emph{avg\_grammar\_rating} : nan\\
\emph{avg\_answerability\_rating} : nan\\
\emph{sum\_yes\_meaningful} : 0\\
\emph{sum\_no\_meaning} : 0\\
\emph{sum\_maybe\_meaning} : 0
\end{quote}

\hypertarget{what-are-beliefs-part-of-the-cause-of}{%
\subsection{What are beliefs part of the cause
of?}\label{what-are-beliefs-part-of-the-cause-of}}

\begin{quote}
\emph{summarized\_paragraph} : Reformed epistemology could be correct
and yet far less significant than its proponents take it to be. That
would occur if in fact rather few religious beliefs are grounded in the
sorts of ordinary religious experiences most believers have. \textbf{For
it may well be that the beliefs are part of the cause of the experience
rather than the other way round.} For example, it could be that most
religious experiences are not the result of religious beliefs, but are
the product of them. It could also be the case that the religious
experience is more important than the religious belief.
\end{quote}

\begin{quote}
\emph{avg\_grammar\_rating} : nan\\
\emph{avg\_answerability\_rating} : nan\\
\emph{sum\_yes\_meaningful} : 0\\
\emph{sum\_no\_meaning} : 0\\
\emph{sum\_maybe\_meaning} : 0
\end{quote}

\hypertarget{how-are-beliefs-more-likely-to-be-part-of-the-cause-of-the-experience}{%
\subsection{How are beliefs more likely to be part of the cause of the
experience?}\label{how-are-beliefs-more-likely-to-be-part-of-the-cause-of-the-experience}}

\begin{quote}
\emph{summarized\_paragraph} : Reformed epistemology could be correct
and yet far less significant than its proponents take it to be. That
would occur if in fact rather few religious beliefs are grounded in the
sorts of ordinary religious experiences most believers have. \textbf{For
it may well be that the beliefs are part of the cause of the experience
rather than the other way round.} For example, it could be that most
religious experiences are not the result of religious beliefs, but are
the product of them. It could also be the case that the religious
experience is more important than the religious belief.
\end{quote}

\begin{quote}
\emph{avg\_grammar\_rating} : nan\\
\emph{avg\_answerability\_rating} : nan\\
\emph{sum\_yes\_meaningful} : 0\\
\emph{sum\_no\_meaning} : 0\\
\emph{sum\_maybe\_meaning} : 0
\end{quote}

\hypertarget{what-can-be-asked-about-faith-in-the-absence-of-belief}{%
\subsection{What can be asked about faith in the absence of
belief?}\label{what-can-be-asked-about-faith-in-the-absence-of-belief}}

\begin{quote}
\emph{summarized\_paragraph} : \textbf{Although the topic is religious
belief the same questions can be asked about faith in the absence of
belief, where the standards might be laxer.} John Schellenberg has
argued that it is not justified to believe in a personal God. Note,
though, that epistemological disputes between Hindu and Buddhist
philosophers over a thousand years ago are much the same as those here
considered. Finally, and more controversially, this entry concentrates
on Western epistemology of religion, which is not, however, the same.
\end{quote}

\begin{quote}
\emph{avg\_grammar\_rating} : nan\\
\emph{avg\_answerability\_rating} : nan\\
\emph{sum\_yes\_meaningful} : 0\\
\emph{sum\_no\_meaning} : 0\\
\emph{sum\_maybe\_meaning} : 0
\end{quote}

\hypertarget{what-type-of-belief-is-the-subject-of-faith}{%
\subsection{What type of belief is the subject of
faith?}\label{what-type-of-belief-is-the-subject-of-faith}}

\begin{quote}
\emph{summarized\_paragraph} : \textbf{Although the topic is religious
belief the same questions can be asked about faith in the absence of
belief, where the standards might be laxer.} John Schellenberg has
argued that it is not justified to believe in a personal God. Note,
though, that epistemological disputes between Hindu and Buddhist
philosophers over a thousand years ago are much the same as those here
considered. Finally, and more controversially, this entry concentrates
on Western epistemology of religion, which is not, however, the same.
\end{quote}

\begin{quote}
\emph{avg\_grammar\_rating} : nan\\
\emph{avg\_answerability\_rating} : nan\\
\emph{sum\_yes\_meaningful} : 0\\
\emph{sum\_no\_meaning} : 0\\
\emph{sum\_maybe\_meaning} : 0
\end{quote}

\hypertarget{what-type-of-beliefs-are-not-the-result-of}{%
\subsection{What type of beliefs are not the result
of?}\label{what-type-of-beliefs-are-not-the-result-of}}

\begin{quote}
\emph{summarized\_paragraph} : Reformed epistemology could be correct
and yet far less significant than its proponents take it to be. That
would occur if in fact rather few religious beliefs are grounded in the
sorts of ordinary religious experiences most believers have. For it may
well be that the beliefs are part of the cause of the experience rather
than the other way round. \textbf{For example, it could be that most
religious experiences are not the result of religious beliefs, but are
the product of them.} It could also be the case that the religious
experience is more important than the religious belief.
\end{quote}

\begin{quote}
\emph{avg\_grammar\_rating} : nan\\
\emph{avg\_answerability\_rating} : nan\\
\emph{sum\_yes\_meaningful} : 0\\
\emph{sum\_no\_meaning} : 0\\
\emph{sum\_maybe\_meaning} : 0
\end{quote}

\hypertarget{what-is-the-cause-of-a-belief}{%
\subsection{What is the cause of a
belief?}\label{what-is-the-cause-of-a-belief}}

\begin{quote}
\emph{summarized\_paragraph} : In what might be called
``counter-reformed epistemology'' it could be allowed that a belief can
be warranted if grounded in a religious tradition. \textbf{Such a belief
would have to be caused in the right sort of way by the right kind of
tradition.} As in the previous cases we might note that such grounding
should be partially accessible to the believer. Rather little work has
been done on this extension of reformed~epistemology, but the social
dimension of warrant has been noted.
\end{quote}

\begin{quote}
\emph{avg\_grammar\_rating} : nan\\
\emph{avg\_answerability\_rating} : nan\\
\emph{sum\_yes\_meaningful} : 0\\
\emph{sum\_no\_meaning} : 0\\
\emph{sum\_maybe\_meaning} : 0
\end{quote}

\hypertarget{what-does-the-autonomy-thesis-say-religious-utterances-are-to-be-judged-by}{%
\subsection{What does the autonomy thesis say religious utterances are
to be judged
by?}\label{what-does-the-autonomy-thesis-say-religious-utterances-are-to-be-judged-by}}

\begin{quote}
\emph{summarized\_paragraph} : Wittgensteinian fideism says that there
are various different ``language games'' and that it is appropriate to
ask questions about justification within a language game. In this way
epistemology is relativised to language games, themselves related to
forms of life, and the one used for assessing religious claims is less
stringent than evidentialism. Here there seems to be both an autonomy
thesis and an incommensurability thesis. \textbf{The autonomy thesis
tells us that religious utterances are only to be judged as justified or
otherwise by the standards implicit in the religious form of life.}
\end{quote}

\begin{quote}
\emph{avg\_grammar\_rating} : nan\\
\emph{avg\_answerability\_rating} : nan\\
\emph{sum\_yes\_meaningful} : 0\\
\emph{sum\_no\_meaning} : 0\\
\emph{sum\_maybe\_meaning} : 0
\end{quote}

\hypertarget{what-are-themselves-justifiably-held-with-full-belief-unless-defeated-by-an-objection}{%
\subsection{What are themselves justifiably held with full belief unless
defeated by an
objection?}\label{what-are-themselves-justifiably-held-with-full-belief-unless-defeated-by-an-objection}}

\begin{quote}
\emph{summarized\_paragraph} : Theistic philosophers may, of course,
grant evidentialism and even grant its hegemony, but defend theism by
providing the case which evidentialists demand. \textbf{To show the
justifiability of full belief that there is a God it is sufficient to
have a deductively valid argument from premisses which are themselves
justifiably held with full belief unless defeated by an objection.} The
details of the arguments are not within the scope of an article on
epistemology. What is of interest is the kind of argument put forward.
\end{quote}

\begin{quote}
\emph{avg\_grammar\_rating} : nan\\
\emph{avg\_answerability\_rating} : nan\\
\emph{sum\_yes\_meaningful} : 0\\
\emph{sum\_no\_meaning} : 0\\
\emph{sum\_maybe\_meaning} : 0
\end{quote}

\hypertarget{is-it-easy-or-easy-to-see-how-disputes-between-two-religions-can-be-seen-as-being-mediated-by-divine-inspiration}{%
\subsection{Is it easy or easy to see how disputes between two religions
can be seen as being mediated by divine
inspiration?}\label{is-it-easy-or-easy-to-see-how-disputes-between-two-religions-can-be-seen-as-being-mediated-by-divine-inspiration}}

\begin{quote}
\emph{summarized\_paragraph} : One obvious complication concerning
religious disagreements is the appeal to divine inspiration, as a source
of private epistemic superiority. \textbf{It is hard to see, though, how
this could apply to disputes between two religions that both rely on the
role of divine inspiration.} Perhaps the only substitute for unargued
dismissal is argued dismissal, as in the case of Martin Luther and the
Protestant Reformed Church of Jesus Christ of Latter-day Saints. The
case for argued dismissal can be summed up as: ``I am not a Christian, I
am a Muslim.''
\end{quote}

\begin{quote}
\emph{avg\_grammar\_rating} : nan\\
\emph{avg\_answerability\_rating} : nan\\
\emph{sum\_yes\_meaningful} : 0\\
\emph{sum\_no\_meaning} : 0\\
\emph{sum\_maybe\_meaning} : 0
\end{quote}

\hypertarget{what-are-language-games-related-to}{%
\subsection{What are language games related
to?}\label{what-are-language-games-related-to}}

\begin{quote}
\emph{summarized\_paragraph} : Wittgensteinian fideism says that there
are various different ``language games'' and that it is appropriate to
ask questions about justification within a language game. \textbf{In
this way epistemology is relativised to language games, themselves
related to forms of life, and the one used for assessing religious
claims is less stringent than evidentialism.} Here there seems to be
both an autonomy thesis and an incommensurability thesis. The autonomy
thesis tells us that religious utterances are only to be judged as
justified or otherwise by the standards implicit in the religious form
of life.
\end{quote}

\begin{quote}
\emph{avg\_grammar\_rating} : nan\\
\emph{avg\_answerability\_rating} : nan\\
\emph{sum\_yes\_meaningful} : 0\\
\emph{sum\_no\_meaning} : 0\\
\emph{sum\_maybe\_meaning} : 0
\end{quote}

\hypertarget{what-type-of-study-is-the-topic-of-this-article}{%
\subsection{What type of study is the topic of this
article?}\label{what-type-of-study-is-the-topic-of-this-article}}

\begin{quote}
\emph{summarized\_paragraph} : Epistemology is confusing because there
are several sorts of items to be evaluated. \textbf{Since the topic of
this article is the epistemology of religion not general epistemological
it will be assumed that what is being evaluated is something related to
faith, namely individual religious beliefs.} The way of evaluating
religious beliefs is as justified or unjustified as the beliefs
themselves. The subject of the article is religion and its role in the
study of religion. The author is a professor of philosophy at the
University of California, Los Angeles.
\end{quote}

\begin{quote}
\emph{avg\_grammar\_rating} : nan\\
\emph{avg\_answerability\_rating} : nan\\
\emph{sum\_yes\_meaningful} : 0\\
\emph{sum\_no\_meaning} : 0\\
\emph{sum\_maybe\_meaning} : 0
\end{quote}

\hypertarget{what-should-be-treated-with-the-same-level-of-respect}{%
\subsection{What should be treated with the same level of
respect?}\label{what-should-be-treated-with-the-same-level-of-respect}}

\begin{quote}
\emph{summarized\_paragraph} : Wittgensteinian fideism would have been
qualified out of existence. All serious intellectual enquiry should also
be treated as parts of the one ``game'', with one set of rules. The
Judeo-Christian-Islamic ``language game'' would be part of this larger,
autonomous metaphysical `` language game'. That modified account would
cohere with the historical fact of the metaphysical commitment of that
religious tradition. \textbf{In that case, though, it would seem that,
not just the Judeo/Christian/Islamic `game', but all serious
intellectual inquiry should be treated with the same level of respect.}
\end{quote}

\begin{quote}
\emph{avg\_grammar\_rating} : nan\\
\emph{avg\_answerability\_rating} : nan\\
\emph{sum\_yes\_meaningful} : 0\\
\emph{sum\_no\_meaning} : 0\\
\emph{sum\_maybe\_meaning} : 0
\end{quote}

\hypertarget{what-belief-is-between-90-and-60}{%
\subsection{What belief is between 90 \% and 60
\%?}\label{what-belief-is-between-90-and-60}}

\begin{quote}
\emph{summarized\_paragraph} : Newman claims that evidentialism falsely
presupposes that there are fine gradations on a scale from full belief
through partial belief to partial disbelief to full disbelief. In such
cases the only available states are those of full belief and full
disbelief or, perhaps, full belief, and lack of full believe. \textbf{Of
course someone can believe that theism has a probability between 90\%
and 60\%, say, but that could be interpreted as believing that relative
to the evidence theism is between 90 and 60\%.} That is a comment on the
strength of the case for theism not the expression of a merely partial
belief.
\end{quote}

\begin{quote}
\emph{avg\_grammar\_rating} : nan\\
\emph{avg\_answerability\_rating} : nan\\
\emph{sum\_yes\_meaningful} : 0\\
\emph{sum\_no\_meaning} : 0\\
\emph{sum\_maybe\_meaning} : 0
\end{quote}

\hypertarget{what-is-the-probability-of-theism}{%
\subsection{What is the probability of
theism?}\label{what-is-the-probability-of-theism}}

\begin{quote}
\emph{summarized\_paragraph} : Newman claims that evidentialism falsely
presupposes that there are fine gradations on a scale from full belief
through partial belief to partial disbelief to full disbelief. In such
cases the only available states are those of full belief and full
disbelief or, perhaps, full belief, and lack of full believe. \textbf{Of
course someone can believe that theism has a probability between 90\%
and 60\%, say, but that could be interpreted as believing that relative
to the evidence theism is between 90 and 60\%.} That is a comment on the
strength of the case for theism not the expression of a merely partial
belief.
\end{quote}

\begin{quote}
\emph{avg\_grammar\_rating} : nan\\
\emph{avg\_answerability\_rating} : nan\\
\emph{sum\_yes\_meaningful} : 0\\
\emph{sum\_no\_meaning} : 0\\
\emph{sum\_maybe\_meaning} : 0
\end{quote}

\hypertarget{what-do-several-religious-traditions-consider-to-be-grounded-in}{%
\subsection{What do several religious traditions consider to be grounded
in?}\label{what-do-several-religious-traditions-consider-to-be-grounded-in}}

\begin{quote}
\emph{summarized\_paragraph} : Reformed epistemology might allow as
warranted those religious beliefs grounded in the event of revelation or
inspiration. Mavrodes has argued that any belief due to a genuine
revelation is warranted. Zagzebski argues that this would have the
unacceptable consequence that warrant, and hence knowledge, becomes
totally inaccessible either to the person concerned or the community.
\textbf{In both cases, the question of whether a belief is genuinely
grounded in religious experience or is genuinely grounding in
inspiration is one that several religious traditions have paid attention
to.}
\end{quote}

\begin{quote}
\emph{avg\_grammar\_rating} : nan\\
\emph{avg\_answerability\_rating} : nan\\
\emph{sum\_yes\_meaningful} : 0\\
\emph{sum\_no\_meaning} : 0\\
\emph{sum\_maybe\_meaning} : 0
\end{quote}

\hypertarget{what-type-of-experience-is-a-belief-grounded-in}{%
\subsection{What type of experience is a belief grounded
in?}\label{what-type-of-experience-is-a-belief-grounded-in}}

\begin{quote}
\emph{summarized\_paragraph} : Reformed epistemology might allow as
warranted those religious beliefs grounded in the event of revelation or
inspiration. Mavrodes has argued that any belief due to a genuine
revelation is warranted. Zagzebski argues that this would have the
unacceptable consequence that warrant, and hence knowledge, becomes
totally inaccessible either to the person concerned or the community.
\textbf{In both cases, the question of whether a belief is genuinely
grounded in religious experience or is genuinely grounding in
inspiration is one that several religious traditions have paid attention
to.}
\end{quote}

\begin{quote}
\emph{avg\_grammar\_rating} : nan\\
\emph{avg\_answerability\_rating} : nan\\
\emph{sum\_yes\_meaningful} : 0\\
\emph{sum\_no\_meaning} : 0\\
\emph{sum\_maybe\_meaning} : 0
\end{quote}

\hypertarget{what-is-the-term-for-a-modification-of-evidential-ism}{%
\subsection{What is the term for a modification of evidential
ism?}\label{what-is-the-term-for-a-modification-of-evidential-ism}}

\begin{quote}
\emph{summarized\_paragraph} : Reformed epistemology might be thought of
as a modification of evidentialism in which the permissible kinds of
evidence are expanded. Notable in this context is Alston's work arguing
that certain kinds of religious experience can be assimilated to
perception. \textbf{Reformed~epistemology~might be thought~of as a~
modification of~ evidential~ism
in~which~permissible~kinds~of~evidence~are~expressed~in a way
that~encourages~perceived~experience.}
\end{quote}

\begin{quote}
\emph{avg\_grammar\_rating} : nan\\
\emph{avg\_answerability\_rating} : nan\\
\emph{sum\_yes\_meaningful} : 0\\
\emph{sum\_no\_meaning} : 0\\
\emph{sum\_maybe\_meaning} : 0
\end{quote}

\hypertarget{what-does-reformed-epistemology-encourage-permissible-kinds-of-evidence-to-encourage}{%
\subsection{What does reformed epistemology encourage permissible kinds
of evidence to
encourage?}\label{what-does-reformed-epistemology-encourage-permissible-kinds-of-evidence-to-encourage}}

\begin{quote}
\emph{summarized\_paragraph} : Reformed epistemology might be thought of
as a modification of evidentialism in which the permissible kinds of
evidence are expanded. Notable in this context is Alston's work arguing
that certain kinds of religious experience can be assimilated to
perception. \textbf{Reformed~epistemology~might be thought~of as a~
modification of~ evidential~ism
in~which~permissible~kinds~of~evidence~are~expressed~in a way
that~encourages~perceived~experience.}
\end{quote}

\begin{quote}
\emph{avg\_grammar\_rating} : nan\\
\emph{avg\_answerability\_rating} : nan\\
\emph{sum\_yes\_meaningful} : 0\\
\emph{sum\_no\_meaning} : 0\\
\emph{sum\_maybe\_meaning} : 0
\end{quote}

\hypertarget{what-is-evidence-that-someone-has-had-a-religious-experience-with-a-certain-content}{%
\subsection{What is evidence that someone has had a religious experience
with a certain
content?}\label{what-is-evidence-that-someone-has-had-a-religious-experience-with-a-certain-content}}

\begin{quote}
\emph{summarized\_paragraph} : Most contemporary epistemology of
religion may be called post modern in the sense of being a reaction to
the Enlightenment, in particular to the thesis of the hegemony of
evidentialism. Evidence may also include the beliefs directly due to
memory and introspection. No beliefs asserting the content of religious
or mystical experiences count as evidence. \textbf{But that does not
prevent the claim that someone has had a religious experience with a
certain content from counting as evidence, even if not treated as
``self-evident''}
\end{quote}

\begin{quote}
\emph{avg\_grammar\_rating} : nan\\
\emph{avg\_answerability\_rating} : nan\\
\emph{sum\_yes\_meaningful} : 0\\
\emph{sum\_no\_meaning} : 0\\
\emph{sum\_maybe\_meaning} : 0
\end{quote}

\hypertarget{what-does-evidence-for-theism-justify}{%
\subsection{What does evidence for theism
justify?}\label{what-does-evidence-for-theism-justify}}

\begin{quote}
\emph{summarized\_paragraph} : Many natural theologians have, however,
abandoned the search for demonstrative arguments, appealing instead to
ones which are probable. \textbf{While there are differences of
approach, the common theme is that there is evidence for theism but
evidence of a probable rather than a conclusive kind, justifying belief
but not full belief.} Notable in this regard are Mitchell's cumulative
argument and Swinburne's Bayesian reliance on probability. In a popular
exposition of his argument SwinBurne appeals instead to an inference to
the best explanation.
\end{quote}

\begin{quote}
\emph{avg\_grammar\_rating} : nan\\
\emph{avg\_answerability\_rating} : nan\\
\emph{sum\_yes\_meaningful} : 0\\
\emph{sum\_no\_meaning} : 0\\
\emph{sum\_maybe\_meaning} : 0
\end{quote}

\hypertarget{what-is-the-main-theme-of-theism}{%
\subsection{What is the main theme of
theism?}\label{what-is-the-main-theme-of-theism}}

\begin{quote}
\emph{summarized\_paragraph} : Many natural theologians have, however,
abandoned the search for demonstrative arguments, appealing instead to
ones which are probable. \textbf{While there are differences of
approach, the common theme is that there is evidence for theism but
evidence of a probable rather than a conclusive kind, justifying belief
but not full belief.} Notable in this regard are Mitchell's cumulative
argument and Swinburne's Bayesian reliance on probability. In a popular
exposition of his argument SwinBurne appeals instead to an inference to
the best explanation.
\end{quote}

\begin{quote}
\emph{avg\_grammar\_rating} : nan\\
\emph{avg\_answerability\_rating} : nan\\
\emph{sum\_yes\_meaningful} : 0\\
\emph{sum\_no\_meaning} : 0\\
\emph{sum\_maybe\_meaning} : 0
\end{quote}

\hypertarget{what-type-of-religious-experience-can-be-assimilated-to-perception}{%
\subsection{What type of religious experience can be assimilated to
perception?}\label{what-type-of-religious-experience-can-be-assimilated-to-perception}}

\begin{quote}
\emph{summarized\_paragraph} : Reformed epistemology might be thought of
as a modification of evidentialism in which the permissible kinds of
evidence are expanded. \textbf{Notable in this context is Alston's work
arguing that certain kinds of religious experience can be assimilated to
perception.} Reformed~epistemology~might be thought~of as a~
modification of~ evidential~ism
in~which~permissible~kinds~of~evidence~are~expressed~in a way
that~encourages~perceived~experience.
\end{quote}

\begin{quote}
\emph{avg\_grammar\_rating} : nan\\
\emph{avg\_answerability\_rating} : nan\\
\emph{sum\_yes\_meaningful} : 0\\
\emph{sum\_no\_meaning} : 0\\
\emph{sum\_maybe\_meaning} : 0
\end{quote}

\hypertarget{what-has-been-stipulated-not-to-be-evidence}{%
\subsection{What has been stipulated not to be
evidence?}\label{what-has-been-stipulated-not-to-be-evidence}}

\begin{quote}
\emph{summarized\_paragraph} : Evidentialism implies that no full
religious belief is justified unless there is conclusive evidence for
it, or it is self-evident. \textbf{The content of religious experience
has been stipulated not to count as evidence.} So the only available
evidence for these beliefs would seem to be non-religious premises. The
only way of deciding whether the religious beliefs are justified would
be to examine various arguments with the non- religious beliefs as
premises and the religious Beliefs as conclusions. This article first
appeared on The Conversation.
\end{quote}

\begin{quote}
\emph{avg\_grammar\_rating} : nan\\
\emph{avg\_answerability\_rating} : nan\\
\emph{sum\_yes\_meaningful} : 0\\
\emph{sum\_no\_meaning} : 0\\
\emph{sum\_maybe\_meaning} : 0
\end{quote}

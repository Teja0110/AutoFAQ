\hypertarget{cancer}{%
\section{\texorpdfstring{\href{https://plato.stanford.edu/entries/cancer/index.html}{Cancer}}{Cancer}}\label{cancer}}

\hypertarget{what-kind-of-hybrid-do-classifications-of-disease-like-cancer-form}{%
\subsection{What kind of hybrid do classifications of disease like
cancer
form?}\label{what-kind-of-hybrid-do-classifications-of-disease-like-cancer-form}}

\begin{quote}
\emph{summarized\_paragraph} : The larger concern cases like this raise
is whether there is one way to privilege choice of causal basis for
classifying diseases like cancer. Different choice of temporal or
spatial scale, or more or less fine-grained characterization of causal
processes might yield multiple, overlapping disease classifications.
\textbf{It seems that classifications of disease like cancer form a kind
of `hybrid'---they are intended to capture natural regularities and
their causes, but also to be of use for a wide array of agents with
different purposes.}
\end{quote}

\begin{quote}
\emph{avg\_grammar\_rating} : 4.3\\
\emph{avg\_answerability\_rating} : 3.7\\
\emph{sum\_yes\_meaningful} : 3\\
\emph{sum\_no\_meaning} : 0\\
\emph{sum\_maybe\_meaning} : 0
\end{quote}

\hypertarget{what-do-all-the-philosophers-seem-to-do-about-the-study-of-mechanisms-associated-with-cancer}{%
\subsection{What do all the philosophers seem to do about the study of
mechanisms associated with
cancer?}\label{what-do-all-the-philosophers-seem-to-do-about-the-study-of-mechanisms-associated-with-cancer}}

\begin{quote}
\emph{summarized\_paragraph} : Philosophical discussions of how and why
TOFT and CSC challenge a simple reductionist picture of cancer etiology
are only two of several examples. Other philosophers have described
evolutionary explanations of cancer (especially models of multilevel
selection) Still others emphasize systems biology-based approaches to
cancer , or the important role of a developmental perspective.
\textbf{What all these philosophers seem to emphasize is that the study
of mechanisms associated with cancer at the cell and molecular level
requires supplementation to better predict and explain the origin of the
disease.}
\end{quote}

\begin{quote}
\emph{avg\_grammar\_rating} : 4.7\\
\emph{avg\_answerability\_rating} : 4.0\\
\emph{sum\_yes\_meaningful} : 1\\
\emph{sum\_no\_meaning} : 2\\
\emph{sum\_maybe\_meaning} : 0
\end{quote}

\hypertarget{what-has-many-causes-and-many-effects}{%
\subsection{What has many causes , and many
effects?}\label{what-has-many-causes-and-many-effects}}

\begin{quote}
\emph{summarized\_paragraph} : The history of attempts to identify
defining features of cancer, let alone arrive at a unified theory, has
floundered. Either cancer is defined so vaguely as to include
non-pathological states, or focused so narrowly on a specific class of
causes. \textbf{Such a variety of definitions is due to the fact that
cancer has many causes, and many effects.} It involves many different
types of dysregulation, at a variety. of temporal and spatial scales. It
is no small challenge to identify defines cancer, and arrive at unified
``theory'' of disease etiology.
\end{quote}

\begin{quote}
\emph{avg\_grammar\_rating} : 4.3\\
\emph{avg\_answerability\_rating} : 5.0\\
\emph{sum\_yes\_meaningful} : 1\\
\emph{sum\_no\_meaning} : 1\\
\emph{sum\_maybe\_meaning} : 1
\end{quote}

\hypertarget{why-is-cancer-a-variety-of-definitions}{%
\subsection{Why is cancer a variety of
definitions?}\label{why-is-cancer-a-variety-of-definitions}}

\begin{quote}
\emph{summarized\_paragraph} : The history of attempts to identify
defining features of cancer, let alone arrive at a unified theory, has
floundered. Either cancer is defined so vaguely as to include
non-pathological states, or focused so narrowly on a specific class of
causes. \textbf{Such a variety of definitions is due to the fact that
cancer has many causes, and many effects.} It involves many different
types of dysregulation, at a variety. of temporal and spatial scales. It
is no small challenge to identify defines cancer, and arrive at unified
``theory'' of disease etiology.
\end{quote}

\begin{quote}
\emph{avg\_grammar\_rating} : 3.7\\
\emph{avg\_answerability\_rating} : 4.3\\
\emph{sum\_yes\_meaningful} : 2\\
\emph{sum\_no\_meaning} : 1\\
\emph{sum\_maybe\_meaning} : 0
\end{quote}

\hypertarget{what-is-the-term-for-the-evolutionary-relationship-between-the-environment-and-the-organism}{%
\subsection{What is the term for the evolutionary relationship between
the environment and the
organism?}\label{what-is-the-term-for-the-evolutionary-relationship-between-the-environment-and-the-organism}}

\begin{quote}
\emph{summarized\_paragraph} : Philosophers of biology have been
skeptical of claims about how our evolutionary history has shaped our
vulnerability to disease. \textbf{Some such claims have been
particularly contentious, either because the nature of the disease
itself is not well characterized or evidence in support of evolutionary
``mismatch'' with our ancestral environment is both scant and disputed.}
Some have argued that many arguments in evolutionary medicine make
``adaptationist'' assumptions, i.e., assumptions that a given trait is
adaptive, or selectively advantageous, founded on at best ``just so''
stories.
\end{quote}

\begin{quote}
\emph{avg\_grammar\_rating} : 5.0\\
\emph{avg\_answerability\_rating} : 1.7\\
\emph{sum\_yes\_meaningful} : 2\\
\emph{sum\_no\_meaning} : 1\\
\emph{sum\_maybe\_meaning} : 0
\end{quote}

\hypertarget{what-kind-of-processes-do-ad-hoc-theories-explain}{%
\subsection{What kind of processes do ad hoc theories
explain?}\label{what-kind-of-processes-do-ad-hoc-theories-explain}}

\begin{quote}
\emph{summarized\_paragraph} : Laplane argues that CSC theory is more
``parsimonious'', because it ``explains cancer development, propagation
and relapse'' from a ``limited number of hypotheses'' In contrast, the
classical view neither predicts nor explains cancer heterogeneity, or
relapse, but must invoke special (``additional'' or ``ad hoc'' theories)
to explain them, in particular, evolutionary processes. Laplane: CSC has
the advantage that it `connects basic research and intervention by
suggesting new therapeutic strategy'
\end{quote}

\begin{quote}
\emph{avg\_grammar\_rating} : 5.0\\
\emph{avg\_answerability\_rating} : 3.7\\
\emph{sum\_yes\_meaningful} : 3\\
\emph{sum\_no\_meaning} : 0\\
\emph{sum\_maybe\_meaning} : 0
\end{quote}

\hypertarget{what-do-explanatory-frameworks-help-generate-strategies-in}{%
\subsection{What do explanatory frameworks help generate strategies
in?}\label{what-do-explanatory-frameworks-help-generate-strategies-in}}

\begin{quote}
\emph{summarized\_paragraph} : Cancer is not a single phenomenon, but a
heterogeneous class of disease processes. This makes not only
classification, but also explanation, enormously difficult, as we will
see below. \textbf{One suggestion that seems plausible is that a better
way of framing the project of cancer research involves commitment to an
``explanatory framework.'' Rather than consisting of sets of laws or
general principles, such frameworks help generate strategies in the
search for causes, and narrow the field of inquiry.} The framework can
``coexist'' or be ``gradually displaced,'' rather than stand in mutually
exclusive relations with one another.
\end{quote}

\begin{quote}
\emph{avg\_grammar\_rating} : 5.0\\
\emph{avg\_answerability\_rating} : 5.0\\
\emph{sum\_yes\_meaningful} : 3\\
\emph{sum\_no\_meaning} : 0\\
\emph{sum\_maybe\_meaning} : 0
\end{quote}

\hypertarget{what-would-the-environment-do-to-cancer-cells}{%
\subsection{What would the environment do to cancer
cells?}\label{what-would-the-environment-do-to-cancer-cells}}

\begin{quote}
\emph{summarized\_paragraph} : The Cancer Genome Atlas project sought to
characterize the genomic features of each cancer type and subtype. While
TCGA did result in a more comprehensive understanding of cancer biology,
it also raised many questions. \textbf{One particular challenge has been
determining which mutations to which genes are the ``drivers'' of
cancer, or which ones enable cancer cells to ``succeed'' in an
environment that would prevent their growth.} Solving this puzzle has
been difficult, because cancers are far more heterogeneous than they
initially anticipated.
\end{quote}

\begin{quote}
\emph{avg\_grammar\_rating} : 4.3\\
\emph{avg\_answerability\_rating} : 2.7\\
\emph{sum\_yes\_meaningful} : 1\\
\emph{sum\_no\_meaning} : 1\\
\emph{sum\_maybe\_meaning} : 1
\end{quote}

\hypertarget{what-are-the-drivers-of-cancer}{%
\subsection{What are the " drivers " of
cancer?}\label{what-are-the-drivers-of-cancer}}

\begin{quote}
\emph{summarized\_paragraph} : The Cancer Genome Atlas project sought to
characterize the genomic features of each cancer type and subtype. While
TCGA did result in a more comprehensive understanding of cancer biology,
it also raised many questions. \textbf{One particular challenge has been
determining which mutations to which genes are the ``drivers'' of
cancer, or which ones enable cancer cells to ``succeed'' in an
environment that would prevent their growth.} Solving this puzzle has
been difficult, because cancers are far more heterogeneous than they
initially anticipated.
\end{quote}

\begin{quote}
\emph{avg\_grammar\_rating} : 4.8\\
\emph{avg\_answerability\_rating} : 3.5\\
\emph{sum\_yes\_meaningful} : 4\\
\emph{sum\_no\_meaning} : 0\\
\emph{sum\_maybe\_meaning} : 0
\end{quote}

\hypertarget{what-theory-cannot-account-for-the-observations-of-tissue-architecture}{%
\subsection{What theory cannot account for the observations of tissue
architecture?}\label{what-theory-cannot-account-for-the-observations-of-tissue-architecture}}

\begin{quote}
\emph{summarized\_paragraph} : Soto and Sonnenschein claim that there
are features of the tissue microenvironment that affect either the
induction of or progression of cancer. Such features are not reducible
to, or explainable in terms of properties or processes at or below the
cell and molecular level, they say. \textbf{Further, they claim that the
somatic mutation theory cannot account for these observations, and that
they can only be accommodated on a view that takes that tissue
architecture itself as causally relevant to the progression of disease.}
\end{quote}

\begin{quote}
\emph{avg\_grammar\_rating} : 5.0\\
\emph{avg\_answerability\_rating} : 5.0\\
\emph{sum\_yes\_meaningful} : 3\\
\emph{sum\_no\_meaning} : 0\\
\emph{sum\_maybe\_meaning} : 0
\end{quote}

\hypertarget{what-type-of-cancer-is-a-major-component-of-the-causal-basis-of}{%
\subsection{What type of cancer is a major component of the causal basis
of?}\label{what-type-of-cancer-is-a-major-component-of-the-causal-basis-of}}

\begin{quote}
\emph{summarized\_paragraph} : Family history and inherited (as opposed
to somatic) mutations, life history (e.g.~parity, or number of children,
and age at first birth), environmental exposures, and histories of
infection, inflammation, and immune response, contribute differentially
to cancer's age of onset, rate of progression, histopathology, and
response to treatment. \textbf{Depending upon how fine-grained a
characterization of the causal basis of cancer one adopts, there could
be hundreds or thousands of cancer types and subtypes.}
\end{quote}

\begin{quote}
\emph{avg\_grammar\_rating} : 1.7\\
\emph{avg\_answerability\_rating} : 2.7\\
\emph{sum\_yes\_meaningful} : 1\\
\emph{sum\_no\_meaning} : 2\\
\emph{sum\_maybe\_meaning} : 0
\end{quote}

\hypertarget{what-is-one-type-of-cancer-type}{%
\subsection{What is one type of cancer
type?}\label{what-is-one-type-of-cancer-type}}

\begin{quote}
\emph{summarized\_paragraph} : Family history and inherited (as opposed
to somatic) mutations, life history (e.g.~parity, or number of children,
and age at first birth), environmental exposures, and histories of
infection, inflammation, and immune response, contribute differentially
to cancer's age of onset, rate of progression, histopathology, and
response to treatment. \textbf{Depending upon how fine-grained a
characterization of the causal basis of cancer one adopts, there could
be hundreds or thousands of cancer types and subtypes.}
\end{quote}

\begin{quote}
\emph{avg\_grammar\_rating} : 2.0\\
\emph{avg\_answerability\_rating} : 1.7\\
\emph{sum\_yes\_meaningful} : 2\\
\emph{sum\_no\_meaning} : 1\\
\emph{sum\_maybe\_meaning} : 0
\end{quote}

\hypertarget{what-do-philosophers-have-on-the-aim-and-character-of-explanation-and-progress-in-cancer-research}{%
\subsection{What do philosophers have on the aim and character of
explanation and progress in cancer
research?}\label{what-do-philosophers-have-on-the-aim-and-character-of-explanation-and-progress-in-cancer-research}}

\begin{quote}
\emph{summarized\_paragraph} : What does it mean to explain cancer? The
question is ambiguous, and fraught. Each of these targets arguably calls
for quite different kinds of explanation. Cancer scientists make appeals
to generalizations of wide scope, for instance, and will also offer
detailed, specific mechanistic explanations of specific outcomes.
\textbf{The fact that these explanations are different in kind may
explain why there appears to be such a variety of views amongst
philosophers concerning the aim and character of explanation and
progress in cancer research.} It's still hotly debated whether all
explanations in the biological sciences must identify causes, or
mechanisms, or whether ``mathematical'' or ``equilibrium'' explanations
are distinct in kind.
\end{quote}

\begin{quote}
\emph{avg\_grammar\_rating} : 2.7\\
\emph{avg\_answerability\_rating} : 4.0\\
\emph{sum\_yes\_meaningful} : 1\\
\emph{sum\_no\_meaning} : 2\\
\emph{sum\_maybe\_meaning} : 0
\end{quote}

\hypertarget{what-are-different-in-kind-in-relation-to-cancer-research}{%
\subsection{What are different in kind in relation to cancer
research?}\label{what-are-different-in-kind-in-relation-to-cancer-research}}

\begin{quote}
\emph{summarized\_paragraph} : What does it mean to explain cancer? The
question is ambiguous, and fraught. Each of these targets arguably calls
for quite different kinds of explanation. Cancer scientists make appeals
to generalizations of wide scope, for instance, and will also offer
detailed, specific mechanistic explanations of specific outcomes.
\textbf{The fact that these explanations are different in kind may
explain why there appears to be such a variety of views amongst
philosophers concerning the aim and character of explanation and
progress in cancer research.} It's still hotly debated whether all
explanations in the biological sciences must identify causes, or
mechanisms, or whether ``mathematical'' or ``equilibrium'' explanations
are distinct in kind.
\end{quote}

\begin{quote}
\emph{avg\_grammar\_rating} : 4.0\\
\emph{avg\_answerability\_rating} : 4.0\\
\emph{sum\_yes\_meaningful} : 1\\
\emph{sum\_no\_meaning} : 2\\
\emph{sum\_maybe\_meaning} : 0
\end{quote}

\hypertarget{what-type-of-causation-do-these-examples-point-to}{%
\subsection{What type of causation do these examples point
to?}\label{what-type-of-causation-do-these-examples-point-to}}

\begin{quote}
\emph{summarized\_paragraph} : Bissell's lab seems to indicate that
structural features of the extracellular matrix somehow either prevent
the emergence of disease, or can accelerate it. \textbf{It seems such
examples do not point to `downward causation,' but instead suggest that
we need to expand our understanding of the causes of cancer to include
maintenance of tissue integrity.} That is, the case of cancer seems to
require a richer, more integrative and interdisciplinary approach to
investigating cancer. It seems each of these authors is pointing to an
important role for structural organization in the maintenance of
homeostasis.
\end{quote}

\begin{quote}
\emph{avg\_grammar\_rating} : 4.7\\
\emph{avg\_answerability\_rating} : 4.0\\
\emph{sum\_yes\_meaningful} : 1\\
\emph{sum\_no\_meaning} : 2\\
\emph{sum\_maybe\_meaning} : 0
\end{quote}

\hypertarget{what-did-a-more-comprehensive-understanding-of-cancer-biology-result-in}{%
\subsection{What did a more comprehensive understanding of cancer
biology result
in?}\label{what-did-a-more-comprehensive-understanding-of-cancer-biology-result-in}}

\begin{quote}
\emph{summarized\_paragraph} : The Cancer Genome Atlas project sought to
characterize the genomic features of each cancer type and subtype.
\textbf{While TCGA did result in a more comprehensive understanding of
cancer biology, it also raised many questions.} One particular challenge
has been determining which mutations to which genes are the ``drivers''
of cancer, or which ones enable cancer cells to ``succeed'' in an
environment that would prevent their growth. Solving this puzzle has
been difficult, because cancers are far more heterogeneous than they
initially anticipated.
\end{quote}

\begin{quote}
\emph{avg\_grammar\_rating} : 5.0\\
\emph{avg\_answerability\_rating} : 4.0\\
\emph{sum\_yes\_meaningful} : 2\\
\emph{sum\_no\_meaning} : 1\\
\emph{sum\_maybe\_meaning} : 0
\end{quote}

\hypertarget{what-type-of-organization-does-each-of-the-authors-point-to-in-the-maintenance-of-homeostasis}{%
\subsection{What type of organization does each of the authors point to
in the maintenance of
homeostasis?}\label{what-type-of-organization-does-each-of-the-authors-point-to-in-the-maintenance-of-homeostasis}}

\begin{quote}
\emph{summarized\_paragraph} : Bissell's lab seems to indicate that
structural features of the extracellular matrix somehow either prevent
the emergence of disease, or can accelerate it. It seems such examples
do not point to `downward causation,' but instead suggest that we need
to expand our understanding of the causes of cancer to include
maintenance of tissue integrity. That is, the case of cancer seems to
require a richer, more integrative and interdisciplinary approach to
investigating cancer. \textbf{It seems each of these authors is pointing
to an important role for structural organization in the maintenance of
homeostasis.}
\end{quote}

\begin{quote}
\emph{avg\_grammar\_rating} : 4.8\\
\emph{avg\_answerability\_rating} : 2.5\\
\emph{sum\_yes\_meaningful} : 4\\
\emph{sum\_no\_meaning} : 0\\
\emph{sum\_maybe\_meaning} : 0
\end{quote}

\hypertarget{is-the-concern-of-whether-there-is-one-way-to-privilege-choice-of-causal-basis-for-classifying-diseases-like-cancer-or-smaller}{%
\subsection{Is the concern of whether there is one way to privilege
choice of causal basis for classifying diseases like cancer or
smaller?}\label{is-the-concern-of-whether-there-is-one-way-to-privilege-choice-of-causal-basis-for-classifying-diseases-like-cancer-or-smaller}}

\begin{quote}
\emph{summarized\_paragraph} : \textbf{The larger concern cases like
this raise is whether there is one way to privilege choice of causal
basis for classifying diseases like cancer.} Different choice of
temporal or spatial scale, or more or less fine-grained characterization
of causal processes might yield multiple, overlapping disease
classifications. It seems that classifications of disease like cancer
form a kind of `hybrid'---they are intended to capture natural
regularities and their causes, but also to be of use for a wide array of
agents with different purposes.
\end{quote}

\begin{quote}
\emph{avg\_grammar\_rating} : 2.0\\
\emph{avg\_answerability\_rating} : 1.7\\
\emph{sum\_yes\_meaningful} : 1\\
\emph{sum\_no\_meaning} : 2\\
\emph{sum\_maybe\_meaning} : 0
\end{quote}

\hypertarget{what-type-of-choice-of-causal-basis-is-there-a-concern-about}{%
\subsection{What type of choice of causal basis is there a concern
about?}\label{what-type-of-choice-of-causal-basis-is-there-a-concern-about}}

\begin{quote}
\emph{summarized\_paragraph} : \textbf{The larger concern cases like
this raise is whether there is one way to privilege choice of causal
basis for classifying diseases like cancer.} Different choice of
temporal or spatial scale, or more or less fine-grained characterization
of causal processes might yield multiple, overlapping disease
classifications. It seems that classifications of disease like cancer
form a kind of `hybrid'---they are intended to capture natural
regularities and their causes, but also to be of use for a wide array of
agents with different purposes.
\end{quote}

\begin{quote}
\emph{avg\_grammar\_rating} : 3.3\\
\emph{avg\_answerability\_rating} : 3.0\\
\emph{sum\_yes\_meaningful} : 2\\
\emph{sum\_no\_meaning} : 1\\
\emph{sum\_maybe\_meaning} : 1
\end{quote}

\hypertarget{what-type-of-model-is-used-to-study-cancer}{%
\subsection{What type of model is used to study
cancer?}\label{what-type-of-model-is-used-to-study-cancer}}

\begin{quote}
\emph{summarized\_paragraph} : Soto and Sonnenschein object not only to
reductionist methodology, but reductionist assumptions about the causes
of cancer. In their view, cancers are not the result of mutations, but
instead ``alterations in the communication among cells and tissues that
affect tissue architecture'' It is these alterations of patterns of
interaction between cells that leads to ``downward causation effects''
on cells and cellular components, inducing aneuploidy and mutations,
they say. \textbf{They bring attention to some interesting broader
questions as to when are we warranted in inferring from cell culture or
model organism work to general claims about cancer etiology.}
\end{quote}

\begin{quote}
\emph{avg\_grammar\_rating} : 5.0\\
\emph{avg\_answerability\_rating} : 2.3\\
\emph{sum\_yes\_meaningful} : 2\\
\emph{sum\_no\_meaning} : 1\\
\emph{sum\_maybe\_meaning} : 0
\end{quote}

\hypertarget{on-what-view-of-cancer-are-cancer-cells-highly-adaptive}{%
\subsection{On what view of cancer are cancer cells highly
adaptive?}\label{on-what-view-of-cancer-are-cancer-cells-highly-adaptive}}

\begin{quote}
\emph{summarized\_paragraph} : Cancer is a disease of aging, perhaps in
part as a byproduct of breakdowns in mechanisms associated with immune
response and tissue integrity. \textbf{On the evolutionary view of
cancer, for instance, cancer cells are in some sense highly adaptive.}
All evolutionary and byproduct explanations of cancer raise similar
philosophical questions about hypothesis testing, as well as definitions
of ``function'' and ``individuality'' Both at the individual and group
level, and over the course of one's life history.
\end{quote}

\begin{quote}
\emph{avg\_grammar\_rating} : 5.0\\
\emph{avg\_answerability\_rating} : 5.0\\
\emph{sum\_yes\_meaningful} : 3\\
\emph{sum\_no\_meaning} : 0\\
\emph{sum\_maybe\_meaning} : 0
\end{quote}

\hypertarget{on-the-evolutionary-view-of-cancer-what-is-it-that-cancer-cells-are-highly-adaptive}{%
\subsection{On the evolutionary view of cancer , what is it that cancer
cells are highly
adaptive?}\label{on-the-evolutionary-view-of-cancer-what-is-it-that-cancer-cells-are-highly-adaptive}}

\begin{quote}
\emph{summarized\_paragraph} : Cancer is a disease of aging, perhaps in
part as a byproduct of breakdowns in mechanisms associated with immune
response and tissue integrity. \textbf{On the evolutionary view of
cancer, for instance, cancer cells are in some sense highly adaptive.}
All evolutionary and byproduct explanations of cancer raise similar
philosophical questions about hypothesis testing, as well as definitions
of ``function'' and ``individuality'' Both at the individual and group
level, and over the course of one's life history.
\end{quote}

\begin{quote}
\emph{avg\_grammar\_rating} : 2.8\\
\emph{avg\_answerability\_rating} : 1.5\\
\emph{sum\_yes\_meaningful} : 1\\
\emph{sum\_no\_meaning} : 3\\
\emph{sum\_maybe\_meaning} : 0
\end{quote}

\hypertarget{what-is-the-term-for-fitness-in-cancer}{%
\subsection{What is the term for fitness in
cancer?}\label{what-is-the-term-for-fitness-in-cancer}}

\begin{quote}
\emph{summarized\_paragraph} : Cancer is a product of breakdown in
mechanisms that permitted the emergence of multicellularity. The
mechanisms that protect us from ``revolt from within'' are not
error-free, and over time will fail. \textbf{On this view, cancer itself
may be viewed as an evolutionary process, where these ``adaptations''
enable short term ``fitness,'' or relative success at survival and
reproduction.} Several scientists have developed theoretical models of
this process, linking it to empirical data, e.g., on emergence of
chemotherapy resistance.
\end{quote}

\begin{quote}
\emph{avg\_grammar\_rating} : 4.0\\
\emph{avg\_answerability\_rating} : 3.0\\
\emph{sum\_yes\_meaningful} : 2\\
\emph{sum\_no\_meaning} : 1\\
\emph{sum\_maybe\_meaning} : 0
\end{quote}

\hypertarget{what-type-of-success-does-cancer-have-at-survival-and-reproduction}{%
\subsection{What type of success does cancer have at survival and
reproduction?}\label{what-type-of-success-does-cancer-have-at-survival-and-reproduction}}

\begin{quote}
\emph{summarized\_paragraph} : Cancer is a product of breakdown in
mechanisms that permitted the emergence of multicellularity. The
mechanisms that protect us from ``revolt from within'' are not
error-free, and over time will fail. \textbf{On this view, cancer itself
may be viewed as an evolutionary process, where these ``adaptations''
enable short term ``fitness,'' or relative success at survival and
reproduction.} Several scientists have developed theoretical models of
this process, linking it to empirical data, e.g., on emergence of
chemotherapy resistance.
\end{quote}

\begin{quote}
\emph{avg\_grammar\_rating} : 4.7\\
\emph{avg\_answerability\_rating} : 4.0\\
\emph{sum\_yes\_meaningful} : 3\\
\emph{sum\_no\_meaning} : 0\\
\emph{sum\_maybe\_meaning} : 0
\end{quote}

\hypertarget{according-to-malaterre-what-is-the-target-of-proportionality}{%
\subsection{According to malaterre , what is the target of
proportionality?}\label{according-to-malaterre-what-is-the-target-of-proportionality}}

\begin{quote}
\emph{summarized\_paragraph} : Malaterre argues that advocates of TOFT
need not commit to non-reductive physicalism, or emergent properties. He
argues that Soto and Sonnenschein are drawing attention to dynamic
relationships between variables that supervene over molecular and
cellular activities. \textbf{Malaterre makes appeal to Woodward's
argument that choice of causal variables often involves appeal to
something like a notion of ``proportionality'' such that, depending on
the target of inquiry, we may be ``led to the incorporation of more
fine-grained detail''}
\end{quote}

\begin{quote}
\emph{avg\_grammar\_rating} : 4.3\\
\emph{avg\_answerability\_rating} : 4.3\\
\emph{sum\_yes\_meaningful} : 1\\
\emph{sum\_no\_meaning} : 2\\
\emph{sum\_maybe\_meaning} : 0
\end{quote}
